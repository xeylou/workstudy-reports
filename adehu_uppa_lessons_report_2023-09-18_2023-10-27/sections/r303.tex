\renewcommand{\figurename}{}
\mychapter{R303 Services réseaux avancés (21h)}{cap:r303}
\lhead{R301 Services réseaux avancés (21h)}

\vspace*{0.2cm}%
      \large
      \href{\@orientadorPagina}{\color{black}Enseignant\\Mr. Laurent Billon}\\%
\vspace*{0.5cm}%

Le module R303 nous a fait abordé deux services réseaux "avancés" que sont le DNS \textit{Domain Name System} et la gestion des mails (envoi et réception). Un cours nous était présenté pour chacun avant de nous atteler à des travaux pratiques pour les mettre en place. L'évaluation étant un QCM de connaissance sur les notions vues en cours et mises en place.

\section{Compréhension de Bind9}

Nous avons étudié le principe de DNS par l'installation du service Bind9, un service de DNS largement utilisé. Nous avons appris par celui-ci les différents types d'enregistrements (A pour les adresses IPv4, AAAA IPv6, CNAME nom symbolique, MX pour mail, comment les agencer...). Nous avons aussi vu comment gérer une zone : enregistrement SOA \textit{Start Of Authority} pour savoir qui a l'autorité sur une zone DNS donnée, la délégation de zone, principe de serveur actif et d'autres passifs pour redondance... J'aborde l'intégralité de ces principes sur un \href{https://xeylou.fr/posts/bind9-workshop/}{article sur mon site} dans un workshop où je montre succinctement les étapes des travaux pratiques.

\section{Mise en place de Postfix et Dovecot}

Nous avons étudié l'implémentation des protocoles de récéption de mail IMAP \textit{Internet Message Access Protocol} et POP3 \textit{Post Office Protocol version 3}. Nous avons aussi abordé le protocole d'envoi de mails SMTP \textit{Simple Mail Transfer Protocol}.
\\ \\
Ces protocoles se font vieillissants mais sont toujours appréciés pour leur qualité : ils font ce qu'ils sont supposés faire. Au fil des années, ceux-ci ont seulement évolués pour s'adapter aux normes de sécurité : STARTTLS, SSL, CRAM MD5... Grands comme petits services, ceux-ci sont toujours largement utilisés, comme Bind9 pour les DNS.
\\ \\
J'aborde une revue complète de l'installation que nous en avons fait \href{https://xeylou.fr/posts/postfix-workshop/}{dans un article sur mon site}.

\section{Aboutissants du module}

Nous avons pu étudier par ce module des protocoles essentiels actuellement dans notre utilisation globale d'internet. Par le montage d'un serveur DNS avec Bind9, nous avons pu pleinement prendre main un architecture que nous avons nous même définie : ce qui est très intéressant pour l'apprentissage. Pareil pour le montage d'un serveur de réception et d'envoi de mails : nous l'avons montés, manipulé, dépanné pour apprendre à connaitre les spécificités des protocols IMAP, POP3 et SMTP pour des manipulations futures.