\renewcommand{\figurename}{}
\mychapter{R3Cyber16 Méthodologie du pentesting (24h)}{cap:r3c16}
\lhead{R3Cyber16 Méthodologie du pentesting (24h)}

\vspace*{0.2cm}%
      \large
      \href{\@orientadorPagina}{\color{black}Enseignant\\Mr. Jean-Jacques Bascou}\\%
      \normalsize
\vspace*{0.5cm}%

Le module R3Cyber16 est le premier module que nous avons eu dédié à la "cybersécurité". Celui-ci abordait les étapes d'un test d'intrusion professionnel lors de ses cours théoriques. Lors des séances de travaux dirigés et pratiques, nous devions comprendre des notions de sécurité d'un SI \textit{système d'information} et leur prévention.
\\ \\
J'aborde en première partie un résumé des enseignements théoriques que j'ai pu avoir lors de ce module. J'interprêtre ensuite ce que nous avons couvert lors des cours dirigiés et des travaux pratiques. J'en conclu par mon impression sur le module et ce que j'en ai à retenir.

\section{Apprentissage théorique des étapes d'un test d'intrusion}

Deux grandes écoles de tests d'intrusion sont prédominantes : la famile OWASP pour le web et les CMS \textit{Content Management System}; \& celle PTEST \textit{Penetration Testing Execution Standard} pour les tests d'intrusion globalisés à un SI ou un groupe de SI.
\\ \\
Dans le cadre de notre formation, nous nous sommes concentrés sur la famille PTEST. Celle-ci définissant un ensemble de normes et de bonnes pratiques, par étapes prédominantes pour un test d'intrusion sur un SI de manière professionnelle.
\\ \\
Un test d'intrusion est caractérisé par un ensemble de processus, qui doivent prouver ou non la sécurité tout ou partie d'un SI, dans des conditions réelles.
\\ \\
Selon le contexte d'intrusion définit au préalable, nous pouvons catégoriser trois types de tests : les tests sous boîtes noires avec aucune connaissance de l'environnement, les tests à boite grise en ayant en partie connaissance de la structure dans laquelle l'attaquant va évoluer et ceux à boîtes blanches avec une connaissance totale de l'environnement.
\\ \\
Chacune de ces familles respectent un ensemble de règles à respecter selon le besoin du test d'intrusion et la criticité du système touché. Ainsi, le PTEST comporte sept grandes étapes que chaque testeur d'intrusion en entreprise se doit de respecter.
\\ \\
La \textbf{période de pré-engagement} définit le périmètre de l'étude (équipements, logiciels, personnels...), avec une date \& une horaire pour l'activité, un contexte global (depuis l'extérieur, l'intérieur...), le niveau de connaissance que doit avoir l'attaquant (vu boîtes - noire l'attaquant ne connaît rien), la limite de son étude (pas les serveurs de sauvegarde) et ses mandats d'autorisation ou des propositions commerciales.
\\ \\
Une fois que l'attaquant a son champs d'action, il doit approfondir sa connaissance du milieu : il doit s'imprégner d'un maximum d'informations extérieures sur sa cible. Cette \textbf{période de reconnaissance} consiste à obtenir le plus d'informations possibles sur le périmètre de son étude (équipements, systèmes d'exploitation, personnel l'utilisant...) par mails bien agencés, analyse systèmes exposés à internet... Ce cadre est encore une fois définit par les boites : blanches - on nous donne tout, grise - on nous donne des informations souvent pour gagner du temps, que l'étude ne revienne moins chère \& noires tout est à faire (HUMINT \textit{Human Intelligence}, OSINT \textit{Open Source Intelligence}...).
\\ \\
Par la suite, après son étude globalisée de la structure, celui-ci va établir une \textbf{modélisation de la menace} \textit{threat modeling}. Cette étape se traduit par une analyse des risques associés à l'activité entreprise en terme de sécurité (accès aux mails, communication entre sites...). Ce qui en résulte en l'étude de la probabilité qu'une menace exploite une vulnérabilité trouvée. Il effectue donc une hiérarchisation du niveau de criticité des services qu'il va toucher, en acceptant les conséquences d'un risque, leur déplacement ou leur limitation (en rapport avec ce qu'il a vu étape de reconnaissance).
\\ \\
La quatrième étape d'\textbf{analyse de vulnérabilité} consiste à rechercher des vulnérabilités exploitables sur une ou plusieurs cibles, plus ou moins critiques sur le threat model. Ces deux étapes peuvent dans certains cas être inversées. L'attaquant va rechercher des vulnérabilités à exploiter sur les SI par différents moyens : reverse-engineering, outils d'OSINT, d'HUMINT, scripts... Il peut s'aider du répertoire de vulnérabilités CVE \textit{Common Vulnerabilty and Exposures} répertoriant les vulnérabilités majeures des services connus et celui de leur criticité CVSS \textit{Common Vulnerability Scoring System} pour indiquer leur niveau de criticité de 1 à 10.
\\ \\
Une fois des vulnérabilités trouvées ou non, l'attaquant va essayer de les exploiter pour arriver à ces fin : \textbf{étape d'exploitation}. Des procédures simples de démonstration d'exploitation de vulnérabilités \textit{proof of concept} existent et sont souvent utilisées. L'attaquant peut aussi exploiter une faille trouvée manuellement, automatiquement par des outils déjà créées, ou concevoir ses propres outils étudiés pour l'exploitation d'une vulnérabilité non répertoriée, trouvée sur le tas \textit{zero day}.
\\ \\
Une fois l'exploitation d'une vulnérabilité réalisée sur un SI, l'attaquant va pouvoir changer son contexte d'attaque en essayant de trouver d'autres machines exploitables depuis la nouvelle machine infectée. Un deuxième objectif est de souvent s'installer de manière pérennisée. Cette étape de \textbf{post-exploitation} se constitue généralement d'une élévation de privilèges sur la machines si besoin, une attaque latérale à d'autres machines ou d'un changement de réseau pour un nouveau découvert sur la machine compromise. Pour ce dernier exemple, il est parfois nécessaire de revenir à l'étape de reconnaissance.
\\ \\
La dernière étape des sept étant le \textbf{rapport} ou le \textit{reporting} des tentatives - les points positifs et négatifs. Elle s'en suit généralement d'une proposition d'amélioration de la sécurité globale du SI pour amener la discussion vers la conclusion sur les risques qui lui ont été demandés de prouver. L'étude essayant de prouver que ceux-ci sont moindres car suffisamment protégés, à les déplacer, ou de les laisser car acceptable. Un attaquant n'ayant pas trouvé de faille sur un SI ne veut pas dire que le SI est bien sécurisé, tout dépend de l'approche / la méthodologie que l'attaquant aborde. Ce n'est que grâce au rapport que vous pourrez conclure que la sécurité de votre SI correspond à vos attentes.

\section{Application et découverte de l'analyse de sécurité}

Lors des travaux pratiques et dirigés de ce module, nous avons été guidé dans la manipulation d'outils pour appliquer succinctement les étapes du modèle PTEST. L'objectif à terme étant de savoir les mettre en lien afin de réaliser un véritable test de pénétration.
\\ \\
Une première partie était dédiée à l'étude de l'étape de reconnaissance. Celle-ci nous montrant des méthodologies applicables via l'OSINT, passant par la recherche d'informations sur le réseau cible depuis internet, allant au regard des enregistrements DNS, ou encore par la récupération automatisée des adresses mails présentes sur le site publique cible...  
\\ \\
Rechercher des informations libres d'accès sur la cible permet d'avoir un point de vue global de la structure, autant informatique qu'humain, pour débuter un plan d'analyse par la suite. Le but étant de réunir un maximum d'informations intéréssantes. Tout ce qui est exposé à internet peut être potentiellement compromis ou être utilisé pour une compromission, même une information mineure.
\\
\textit{dig, whois, dnsrecon, dnsenum, nmap, Spiderfoot, The Harvester, dirb utilisés}
\\ \\
La deuxième étape de recherche de vulnérabilités a aussi était couverte par l'analyse plus approfondie du réseau (analyse de ports aggressif, test utilisation protocole SNMP, Openvas pour les scans de SI \& permet de faire remonter un rapport de vulnérabilités)... Cette partie étant l'une des plus techniques, nécessitant une bonne source de connaissance en informatique et en réseau dans un lapse de temps relativement cours. Elle nécessite aussi énormément de savoir manipuler ses outils, une courbe d'apprentissage plus ou moins abbrute peut être ressentie.
\\
\textit{netdiscover, nmap, snmp-check, Openvas, Greenbone utilisés}
\\ \\
Nous avons vu transversalement les attaques par force brute \textit{bruteforce} et par dictionnaire. Cas d'exemple - certains utilisateurs peuvent utiliser le nom de leur animal, de leur enfant, avec une suite de nombre correspondant à une date symbolique, comme un mariage ou une naissance, pour définir leur mot de passe. Avec les informations récupérées sur les personnes et sur l'infrastructure informatique lors de l'étape de reconnaissance, et en connaissance que la cible n'est pas un utilisateur avancé d'informatique : nous pouvons essayer de générer une suite de mot de passe contenant les informations de la victime réagencées pour essayer de trouver son mot de passe sur une machine (Milou235182, Pipou!12/24...).
\\ \\
Cette méthode ayant une limite : le nombre de tentative est très souvent limité. Plus loin que d'essayer de trouver un mot de passe, nous pouvons essayer de comparer la signature du mot de passe (car il est forcément stocké) à cette liste. Les mots de passe ne sont généralement pas enregistrés en clair - si quelqu'un ouvrait le fichier contenant votre mot de passe, il ne pourrait pas le lire. À la place, ceux-ci sont renseignés par empreinte - fonctions mathématiques complexes ne permettant pas de retrouver la source. Des listes d'empreintes sont générées par algorithme de chiffrement, si l'attaquant réussi à récupérer l'empreinte d'un mot de passe \textit{hash} (car stocké quelque part) : il peut le comparer chez lui à une liste de hashs \textit{rainbow list} de mots de passe connus pour essayer de le retrouver, pour peu que votre mot de passe soit courant. Il est aussi possible de "cracker" un mot de passe, mais demandant une force de calcul impressionnante, déjà que la comparaison en demandant beaucoup.
\\
\textit{Crunch, Cewl \& Hydra pour génération de dictionnaire, Hashcat \& John the ripper pour comparaisons}
\\ \\
La dernière étape étudiée fut celle de l'exploitation de vulnérabilités, majoritairement avec Metasploit. Metasploit est un outil permettant de regrouper des exploitations de vulnérabilités connues, c'est une grande boite à outils, un framework... Un bon attaquant, par ses connaissances, devra parfois créer ses propres outils pour ses tests d'intrusion, car utilisation trop spécifique.
\\ \\
Le cheminement d'utilisation de Metasploit se résumé à : rechercher une vulnérabilité (numéro de version d'un logiciel, protocole non sécurisé...), selectionner un \textit{exploit} - un programme dans Metasploit renseigné pour exploiter cette vulénrabilité, et changer son payload (contenu du programme qui est malveillant) pour le remplacer par les informations nécessaires - peuvant être le changement d'une adresse IP, certains paramètres...
\\ \\
L'utilisation de Metasploit est extrèmement intéressante pour beaucoup de cas d'usages, mais il faut tout de même avoir une bonne base théorique des protocoles étudiés pour les appliqués. Aussi avoir de bonnes notions en réseau pour les utiliser correctement. Nous sommes aussi limités par ce qui est présent dans Metasploit, sa courbe d'apprentissage est aussi extrêmement très rapide car simple mais vite limitée par ce qu'il propose, plutôt que de nous faire une collection de nos propres outils pour exploiter les vulnérabilités qu'on trouve.
\\
\textit{montage d'un reverse shell php, utilisation exploit ssh version 1, exploitation/proof of concept log4j par renseignement cve, scan réseau avec Metasploit par arp sweep, élévation de privilège, évasion de conteneur (docker escape), msfvenom, lancement d'un reverse shell par crontab}

\section{Aboutissants de l'enseignement}

Ce module nous a permis de comprendre la méthodologie d'un test d'intrusion professionnel. Nous avons vu un exemple d'application des étapes de l'école PTEST, par l'accompagnement dans l'utilisation d'outils dédiés. Reste à nous de définir nos propres méthodologies, de reprendre, ou s'inspirer de celles vues en cours.
\\ \\
Je retiens que pour un test d'intrusion, une grande connaissance théorique est nécessaire pour espérer en faire un bon : notamment pour rendre un rapport cohérent... Il ne suffit pas de copier-coller des démonstrations d'outils trouvés sur internet, ou encore d'ouvrir Metasploit et espérer faire fonctionner le bon exploit... Apprendre en même temps de découvrir ce qu'on a sous les yeux est possible mais dangereux...