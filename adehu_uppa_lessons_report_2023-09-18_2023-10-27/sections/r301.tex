\renewcommand{\figurename}{}
\mychapter{R301 Réseaux de campus (30h)}{cap:r301}
\lhead{R301 Réseaux de campus (30h)}

\vspace*{0.2cm}%
      \large
      \href{\@orientadorPagina}{\color{black}Enseignant\\Mr. Shidoush Siami}\\%
      \normalsize
\vspace*{0.5cm}%

Le module R301 avait pour objectif de nous faire manipuler les technologies utilisées dans les réseaux de campus. Sont définis des réseaux de campus de grandes infrastructures informatiques d'entreprise, généralement multi-sites ou complexes. Ainsi, les technologies abordées touchent aux domaines de l'automatisation de tâches - pour prévenir la répétition d'activités manuelles sur un grand nombre de postes -, la gestion de SI à distance - éviter les déplacements fréquents \& favoriser la gestion dynamique d'un réseau -, des tables de routage dynamique, et la redondance de routeur de façade.
\\ \\
Nous avons plus spécifiquement vu dans ce module un renforcement de l'utilisation du protocole SSH \textit{Secure Shell} par l'utilisation de clés asymétriques, le déploiement d'actions simple par SSH avec Ansible, une introduction au routage inter-VLAN, et enfin une utilisation des protocoles OSPF, HSRP, EIGRP \& BGP.

\section{Hardening du protocole SSH}

Nous avons utilisé le protocole SSH par le passé à l'IUT sans l'avoir étudié. Cette fois-ci, nous l'avons étudié et fait du hardening \textit{renforcement} en intégrant l'utilisation de clés plutôt que de mots de passe pour l'initiation d'une session.
\\ \\
Au lieu d'utiliser des mots de passe comme méthode d'authentification, le protocole SSH permet d'utiliser un couple de clé publique \& privée pour s'authentifier. Les clés étant plus robustes au déchiffrement par leur longueur plus importante et aléatoire (1024 bits, 2048...). L'utilisateur peut aussi ne pas connaître le mot de passe de sa session Shell, pour lui restreindre dans ses activités, pas sa connexion.
\\ \\
Les premières séances de travaux pratiques étaient dédiés à la création et l'utilisation des clés sur des machines GNU/Linux. J'ai rajouté une partie pour les intégrer à des équipements Cisco, non présent sur le cours de base, rajouté par la suite. Accessible sur \href{https://xeylou.fr/posts/ssh-cisco-ios/}{mon site}.

\section{Revu des possibilités d'Ansible}

SSH est un outil extrêmement utile et largement utilisé pour l'administration de systèmes à distance. Nous avons précédemment vu que son utilisation pouvait être renforcée par un chiffrement asymétrique plutôt que la comparaison de hash \textit{signature} de mots de passe. Mise à part que SSH est un outil extrêmement puissant, celui-ci peut devenir complexe à maintenir pour effectuer de grandes suites d'opérations sur un parc hôtes (lancer des commandes à chaque fois...).
\\ \\
Pour exécuter une commande sur un parc de machine, j'avais l'habitude de générer des scripts Shell \& de lancer les commandes SSH que je souhaitais faire vivre dedans. Je pouvais faire exécuter ce script par un crontab \textit{planificateur de tâches sur GNU/Linux} pour les lancer périodiquement. Une solution plus adaptée et flexible est disponible : Ansible. Celui-ci permet l'automatisation de tâches sur tout type de systèmes d'exploitations, dépendemment du protocole contacté. Il facilite le déploiement de procédure par SSH, le rendant plus professionnel. Utile pour effectuer des vérifications de sauvegardes périodiques, une initiation de sauvegardes etc. Ansible se base sur le format de fichier YAML \textit{Yet Another Markup Language} pour déclarer des services à déployer (vérification de [...], installation de [...] etc.).
\\ \\
Ansible utilise des playbooks pour déclarer les actions qu'il va effectuer. Cette solution étant très flexible, nous n'avons exploité qu'une infime partie de ses capacités en nous restreignant à l'utilisation de SSH pour les playbooks. Nous avons exécutés des commandes SSH sur des hôtes définis par leurs adresses IP ou noms symboliques pour montrer une utilisation propre de gestion d'un parc informatique.

\section{Types de routage inter-VLAN}

Une séance de travail était dédiée à la manipulation du routage de VLANs \textit{trâmes IP tagguées} : le routage inter-VLAN. Un mode de routage simple a été vu pour les router à l'instar des réseaux physiques, en utilisant un lien physique par VLAN. Une interface du routeur ayant une adresse IP sur un VLAN. Ce principe montrait ses limites pour un grand nombre de VLAN : pour 100+ VLAN, besoin de 100+ câbles et interfaces sur un routeur (impossible). 
\\ \\
Nous avons donc abordé une alternative à cette méthode, grace à une particularité des VLANs. Ceux-ci peuvent être transportés sur un lien sans se voir : en utilisant un lien trunk. Si il y a utilsation d'un lien trunk jusqu'au routeur, alors celui-ci peut diviser son l'interface sur ce lien pour accepter uniquement des trames tagguées pour un VLAN particulier et déclarer une adresse IP sur celui-ci. Ainsi ont été abordées les sous-interfaces par VLAN, le routeur possédant une adresse IP par sous-interface, elle-même étant attribuée à un VLAN. Ce qui permet au routeur de connaître plusieurs réseaux caractérisés par leur VLAN sur un lien, et de pouvoir les router.
\\ \\
Ce routage inter-VLAN on stick a été vu, supprimant la contrainte du nombre de câbles imposés par un grand nombre de VLANs en abordant un lien trunk entre un switch et un routeur. Ce lien trunk n'acceptant que les VLANs devant être routés, les sous-interfaces restent mais pour la même interface physique.
\\ \\
J'aborde le routage inter-VLAN "on-stick" (utilisation d'un lien trunk) et celui obsolète (une interface, un VLAN) sur \href{http://xeylou.fr/posts/cisco-inter-VLAN/}{un article sur mon site}.

\section{Étude de l'OSPF}

Nous avons abordés deux protocoles de campus pour la gestion de routeurs, orientés Cisco. Un premier pour la gestion des tables des routage de manière dynamique, l'OSPF \textit{Open Shortest Path First}. Puis un autre pour instaurer une redondance de passerelle dans un réseau local, l'HSRP \textit{Hot Standby Router Protocol} (propriétaire Cisco).
\\ \\
Les routeurs définissent les routes entre les réseaux qu'ils raccordent. Les réseaux de campus, à l'instar de grands réseaux d'entreprises multi-sites ou de datacenters \textit{centre de données}, ont des réseaux comportant une multitude de routeurs à faire communiquer pour faire joindre leurs campus. Pour définir le chemin que doivent prendre les informations pour atteindre leur destination, les routeurs ont connaissance des routes pour rediriger les informations vers leur réseau cible. Ces routes sont répertoriées dans une table de routage, individuelle à chaque routeur, pour rerouter les paquets vers les réseaux de destination. Les tables de routage ont un modèle de maintenance manuel par défaut, cette politique devient un problème prépondérant pour l'administration d'un très grands nombre d'appareils (lors de changement de routes, on doit annoncer sur tous les routeurs le changement de route manuellement; c'est long et répétitif - ce qui augmente la possibilité d'inadvertance, pareil si un lien tombe en panne : on doit redéfinir les routes).
\\ \\
Afin de gérer les routes des paquets pour des réseaux avec un grand nombre de routeurs, les \textbf{protocoles de routage} ont commencés à voir le jour pour automatiser la création de routes et leur distribution. Le fonctionnement de ces protocoles varient, mais une méthodologie reste en place : un nouveau réseau est déclaré sur un routeur, ce routeur communique avec ses autres routeurs du parc pour annoncer son nouveau réseau, ceux-ci l'ajoutent \& se créent leur règle dans leur table de routage pour permettre l'inter-connexion du nouveau réseau avec ceux déjà existants (chacun peut avoir une route différente selon l'issue de l'algorithme). Tout cela fonctionne uniquement si les routeurs communiquent avec le même protocole, car utilisant les mêmes algorithmes (ensemble de procédures) pour travailler.
\\ \\
Dans cette lignée, l'algorithme de Dijkstra créée en 1959 par Edsger Dijkstra, cherchant le chemin le plus court entre deux points d'un graphique de points selon le coup d'utilisation des liens, fut les prémices de ces protocoles de routage. L'algorithme reposant sur la théorie des graphs pour les mathématiques, l'une des applications trouvées fut pour le routage des informations en informatique. Ainsi, OSPF est une application de cet algorithme pour créer un protocole permettant, à la déclaration d'un réseau, de trouver le meilleur chemin pour le rejoindre aux réseaux déjà renseignés selon l'état des liens existants (débit d'un lien, liens brisés...). Les routeurs se partagent leur table de routage périodiquement pour apprendre de nouvelles routes. À l'apprentissage d'une nouvelle route, le protocole OSPF fait appel à l'algorithme de Dijkstra pour concevoir la route la optimale pour joindre ce nouveau réseau.
\\ \\
L'inventivité derrière l'algorithme de Dijkstra, la conceptualisation du protocole OSPF et son implémentation éléctronique pour les appareils furent un vrai travail. Cependant, sa configuration sur les équipements est devenu extrêmement simple : en donnant une suite de commandes, l'équipement sait qu'il doit utiliser le protocole et dans quelles circonstances. En quelques autres, nous ajoutons un réseau à la zone de routeurs. Nous pouvons définir l'intervale de temps à laquelle les routeurs se partagent leurs informations de routage, les types de message envoyés etc.
\\ \\
Ce module demandait de configurer ses protocoles sur des équipements Cisco sur Packet Tracer et dans un cas d'usage physique. Nous n'avons pas abordé ce que j'ai abordé sur l'aspect théorique des protocoles. Encore une fois, j'ai rédigé \href{http://xeylou.fr/posts/cisco-ospf/}{un article sur mon site sur l'OSPF}, abordant son implémentation sur routeurs Cisco.

\section{Passage sur HSRP}

Le deuxième protocole de campus étudié fut HSRP. Chaque réseau informatique possède une passerelle si celui-ci souhaiter accéder à d'autres réseaux (internet, un campus distant...). Les passerelles sont un élément extrêment important pour la vie d'un réseau, souvent caractérisées par des routeurs, ceux-ci définissent les routes à prendre par les paquets (une route pour internet, une autre pour le réseau du site distant...). Si la passerelle vient à ne plus jouer son rôle, un réseau peut se retrouver isolé à ne plus pouvoir communiquer avec l'extérieur.
\\ \\
HSRP intervient en permettant la redondance d'une passerelle en déportant le point unique de défaillance qu'est le routeur d'un site en deux ou plus. On divise donc en théorie par deux la possibilité d'avoir un problème de survenant sur passerelle (si correctement exploité, dépendemment du problème). Extrêmement demandé sur de grands réseaux d'entreprises ou de campus, où l'on ne peut se permettre de ne pas permettre le travail d'une centaine ou plus de personnes.
\\ \\
Ce protocole permet de lier deux routeurs et de se les faire partager une adresse IP virtuelle, qui sera la passerelle. Les équipements du réseau communiqueront donc avec la passerelle par son adresse IP virtuelle, ne sachant quel équipement est derrière. Les routeurs synchronisent leur activité (table de routage...) pour avoir la même configuration sur chacun. Pendant la configuration du protocole, l'administrateur réseau défini un routeur qui sera celui principal, et un ou plusieurs autres secondaires. Ainsi est défini un ordre de priorité par les routeurs : si le principal ne répond plus, celui avec la priorité la plus haute le suivant prendra l'adresse IP virtuelle pour devenir la passerelle du réseau; pour ainsi garder une continuité dans les activités du réseau.
\\ \\
Je n'aborde pas la configuration des routeurs, l'ayant fait \& expliqué sur \href{http://xeylou.fr/posts/cisco-hsrp/}{mon site}.

\section{Activités EIGRP et BGP}

La dernière partie du module était facultative, celle-ci abordée les protocoles de routage EIGRP et BGP. Les activités en rapport n'étaient pas demandées à l'examen. Dans cette optique, nous avons pu utiliser la dernière séance de travail pour réviser certaines parties du modules avant l'examen ou la consacrer à l'étude de ces deux protocoles (ce que j'ai décidé de faire).
\\ \\
EIGRP \textit{Enhanced Interior Gateway Routing Protocol} est un protocole de routage au même titre qu'OSPF. Celui-ci se base sur l'algorithme DUAL \textit{Diffusing Update Algorithm}, et est propriétaire de Cisco, utilisable seulement sur leurs appareils - à contrario d'OSPF. Les deux utilisent des messages hello pour se transmettre leurs voisins. Deux différences majeures peuvent être notifiées : EIGRP permet l'équilibrage de charge sur des liens à débits inégaux, et OSPF calcule son coût uniquement via la bande passante des liens contre EIGRP qui le calcule selon la charge, la bande passante, le délai et la "fiabilité" d'un lien (ça ne sert à rien d'avoir un lien à 10 Gbits/s si il n'est efficace que 1\% du temps).
\\ \\
Nous n'avons pas abordé ce qui a été mentionné dans le paragraphe au dessus, uniquement l'implémentation du protocole sur routeurs Cisco. Son implémentation reste extrêmement simplifiée - besoin de comprendre et d'utiliser quelques commandes.
\\ \\
BGP \textit{Border Gateway Protocol} est un protocole de routage dynamique comme EIGRP et OSPF. Son utilisation en reste cependant différente, extrêmement utilisé au quotidien pour déclarer des réseaux sur internet. Les entreprises achètent des groupes d'adresses IP publiques, avec des masques de sous-réseau (/24, /32...) : des AS \textit{Autonomous System}. Ces AS doivent être rattachés à des routeurs pour être déclarées sur internet. BGP s'assure de faire communiquer les différentes AS existantes. Ainsi, nous avons vu comment joindre deux AS dans un lab Cisco en utilisant BGP. Avec plus de recherches, nous pouvons aussi faire de l'IBGP à l'intérieur d'une AS par exemple.

\section{Aboutissants du module}

Beaucoup de notions importantes des réseaux ont été abordés par ce module, particulièrement des réseaux de campus. Ainsi, nous avons pu étudier pleinement et mettre en place un système d'administration de machines d'un grand parc par Ansible / SSH + clés. Nous avons aussi vu l'interconnexion de routeurs à grande échelle par des protocoles toujours utilisés aujourd'hui : BGP, OSPF, EIGRP... D'autres implémentations cruciales ont aussi été vues dans notre enseignement, je pense au routage inter-vlan omniprésent et la redondance de passerelle avec HSRP.