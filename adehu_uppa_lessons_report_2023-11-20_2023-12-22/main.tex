\documentclass[tcc]{ic}

\hypersetup{
    colorlinks = {true},
    linktocpage = {false},
    plainpages = {false},
    linkcolor = {Blue},
    citecolor = {Blue},
    urlcolor = {Red},
    unicode = {true},
    pdftitle = {Alexis Rapport des enseignements à l'IUT UPPA du 20-11-2023 au 22-12-2023}, 
    pdfauthor = {Alexis Déhu},
    pdfsubject = {Rapport de mes enseignements reçus à l'IUT de Mont-de-Marsan pour la deuxième période de ma deuxième année de BUT R&T du 20-11-2023 au 22-12-2023},
    pdfkeywords = {enseignements, apprentissage, alternance, iut, but, rt, uppa, pau, mont, de, marsan, mont-de-marsan}
}

\usepackage{algorithm}
\usepackage{algpseudocode}

\makeatletter
\renewcommand{\ALG@name}{Algoritmo}
\renewcommand{\listalgorithmname}{List of \ALG@name s}
\makeatother

\newcolumntype{Y}{>{\centering\arraybackslash}X}

\begin{document}

    % Inclui o preâmbulo do documento (Informações da capa e contracapa)
    \titulo{Rapport des enseignements à l’IUT}

\autor{Alexis Déhu}{a.dehu@aditu.fr}{}


\orientador{M. Angel Abénia}{abenia@univ-pau.fr}{}{}{}
%\examinador{Mr. Guillaume Devesa}{g.devesa@aditu.fr}{}{}{}

\dataMesAno{Septembre}{2023}{}

    \selectlanguage{english}
    
    % Capa, contracapa e avaliadores
    \capa

    % \aprovacao
    
    % Agradecimentos
    % \begin{agradecimentos}

Obrigado

\end{agradecimentos}
    
    % Resumo e Abstract
    \begin{resumo}
    \textit{blabla}
    \\ \\
    \textit{blabla}
    
    
    \palavrasChave{recherche de solutions; autonomie; compréhension des besoins; documentation; déploiement d'applicatif; rédaction; travaux de recherche}
\end{resumo}
    %% \begin{abstract}
%     Lorem ipsum dolor sit amet, consectetur adipiscing elit. Duis elit tellus, vehicula in justo eget, fermentum aliquet nisi. In id quam mauris. Sed id vehicula libero. Quisque id tortor placerat, consequat ex sed, placerat ante. Ut dui nunc, placerat volutpat ipsum sit amet, dictum pellentesque lacus. Donec leo sem, dictum quis lacus at, ultricies sagittis elit. Mauris sit amet tortor efficitur, luctus justo sit amet, tincidunt tellus. Donec vulputate non risus at tincidunt. Sed accumsan at erat in aliquet. Sed consequat gravida bibendum. Sed fermentum metus sed ex lacinia mattis. Phasellus vel enim nisl. Proin tortor dui, luctus in erat a, dapibus convallis turpis. Maecenas vitae vulputate neque.

%     Etiam ac ante a lorem consectetur varius. Donec vitae dui porttitor, efficitur ante sed, aliquet lectus. Etiam aliquet mattis sagittis. Integer accumsan, est nec vehicula suscipit, nibh nisl varius ante, at convallis urna est dapibus massa. Orci varius natoque penatibus et magnis dis parturient montes, nascetur ridiculus mus. Praesent tempus dolor sit amet metus eleifend porta. Vestibulum eget viverra nulla. Mauris ac condimentum augue, quis molestie tortor. Duis ac condimentum tortor, sed ullamcorper nunc. Vivamus et suscipit arcu. Duis eget rutrum est, sed pellentesque leo. Nulla lorem lacus, faucibus vitae lectus eu, porttitor efficitur ante. Sed risus tortor, venenatis quis convallis non, blandit in orci. Morbi et neque hendrerit, consectetur magna vitae, consectetur dolor. Nam nec iaculis urna, vitae tincidunt eros. Ut pretium neque convallis turpis rutrum, nec dapibus est faucibus.
    
%     Quisque id laoreet ligula. In hac habitasse platea dictumst. Fusce scelerisque, nunc non lacinia maximus, lorem metus rutrum lacus, ac maximus dolor tellus at velit. Sed diam leo, interdum ut sapien at, finibus aliquam diam. Praesent vel erat id enim scelerisque fringilla. Aliquam cursus risus vulputate ex interdum, sit amet pretium augue semper. Curabitur eget risus eget nulla placerat ornare. Nam quis ornare ipsum. Duis id feugiat lectus. Phasellus vehicula leo id consequat porta. Aliquam ut massa malesuada, ultricies ipsum nec, euismod eros. In lacinia aliquet leo non mattis. Etiam semper neque risus, a condimentum diam euismod eget. Suspendisse vulputate viverra mauris, ac sollicitudin tellus placerat non. Nullam felis metus, congue sed neque ac, iaculis sollicitudin nisl.
    
%     \keywords{graphics processing; medical images; computer vision; deep learning; data augmentation;}
% \end{abstract}
    
    % Sumário
    \renewcommand*\contentsname{Table des matières}
    \tableofcontents
    \thispagestyle{empty}
    
    % Início dos capítulos
    \inicio
    
    \renewcommand{\figurename}{}
\mychapter{R3.ROM16 Ingénierie de la téléphonie sur IP (22h30)}{cap:r3rom16} 
\lhead{R3.ROM16 Ingénierie de la téléphonie sur IP (22h30)}

\vspace*{0.2cm}%
      \large
      \href{}{\color{black}Enseignant\\M. Angel Abénia}\\%
      \normalsize
\vspace*{0.5cm}%

Ce module accès sur la ToIP \textit{Téléphonie sur IP} fait suite aux enseignements reçus en première année sur VoIP \textit{Voix sur IP}. Nous y avons revu la numérisation de la voix, pour y ajouter les technologies logicielles et réseaux permettant son transport et leur mise en place en entreprise.
\\ \\
Ainsi, nous y avons étudié les codecs, le framing et la VAD pour basculer la donnée analogique et continue qu'est la voix dans le monde du numérique. Nous y avons abordé son transport sur le réseau avec les protocoles RTP et RTCP. Pour finaliser notre apprentissage par le montage d'une infrastructure ToIP afin d'analyser la signalitique utilisée pour l'émission et la gestion des appels en entreprise de nos jours.

\section{Capture de la voix vers le numérique}
% trois types de codecs
% codecs G721, G711
% pas précisé codec pour codeur décodeur

La voix est une donnée analogique, c'est un flux constant d'informations matérialisé par la vibration de l'air, retranscrit dans nos tympans et interprêté par notre cerveau. Nous arrivons à la numériser, de la passer du monde analogique (phénomène physique du son) à celui du numérique (de l'informatique, avec des nombres).
\\ \\
Cette conversion analogique-numérique débute par l'échantillonnage de la voix : on place une membrane pour reproduire artifitionnellement nos tympans dans un micro, pour poser une valeur numérique sur la vibration que la membrane reçoit.

% \begin{figure}[H]
\begin{figure}[hp]
    \centering
    \includegraphics[width=\textwidth - \textwidth / 5]{ressources/r3rom16/00.png}
    \caption{La vibration reçue s'apparente à une fréquence, l'échantilloneur pose 8000 points par secondes dessus pour avoir une voix échantillonnée à 8 KHz et le codeur définit le nombre d'étages que ces points peuvent prendre en bits multiples de 2}
    \label{fig:echantillonnage}
\end{figure}

\noindent En jouant sur ces paramètres, on peut retrouver une voix plus ou moins fidèle à l'analogique car plus d'échantillons seront pris pour caractériser la voix (échantillonnage), ou plus ceux-ci pourront être variés car plus d'étages (codage). Une compression est souvent présente, car chaque étage correspond à une valeur chiffrée multiple de 2 (2 bits, 4 bits...) et l'échantillonnage définira la fréquence à laquelle on souhaitera générer des nombres avec ces valeurs (pouvant vite prendre de la place).
\\ \\
La compression se retrouve extrêmement utile pour stocker la voix et la transmettre sur le réseau. Les ensembles échantillonnage, codage et compression ont été normalisés pour que les téléphones et autres appareils audio puissent enregistrer et rejouer les mêmes enregistrements. Ainsi sont nés les codecs audio.
\\ \\
Ceux-ci sont extrêmement importants en réseau car définissent la quantité de données envoyés (rappel, elles prennent de la place). Les deux codecs les plus utilisées sont le G 711 et le G 729. Le G 711 propose 8 KHz d'échantillonnage sur 8 bits avec 0,125 ms de temps de compression : 64 kbits/s seront envoyés sur le réseau (prendra davantage de bande passante que le G 711 avec ses 8 kbits/s).
\\ \\
Nous avons étudié les codecs, le framing, la VAD impactant le stockage et le transport de la voix sur les réseaux. Ainsi, nous pourrons efficacement comprendre, dimmensionner et corriger des problèmes impactant la téléphonie sur IP d'une entreprise.

\section{Transporter une information continue sur un réseau}
% problème des applications temps réels
% RTCP, RTP

Le transport de la voix sur le réseau impose des problématiques non rencontrés auparavant par les flux de données (téléchargement de fichiers, vidéo à la demande...). Une conversation orale ne peut se permettre de perdre un bout de l'information, encore moins de la redemander à chaque fois, que l'on vérifie à chaque fois que l'information soit bien arrivée, ou que les morceaux de la phrase n'arrivent pas en même temps.
\\ \\
Pour compenser ces problèmes instaurés par l'utilisation de réseaux à commutation de paquets comme IP, un système de priorisation été mis en place pour ceux l'utilisant : la QoS \textit{Qualité de Service}. Ainsi, les paquets comportant de la voix seront privilégiés en étant routés ou commutés plus vites que les autres (pour éviter les latences, les gigues...).
\\ \\
Des protocoles de couche 4 ont été étudiés pour le transport de la voix dit \textit{"temps réel"}, et toujours largement utilisés : RTP et RTCP. Nous comprennons désormais les téchnologies permettant de transmettre la voix sur le réseau.

\section{Comment les appels sont passés en ToIP}
% signalisation SIP

Les signalisation téléphoniques sur la ToIP ont été standardisés afin de pouvoir initier, manipuler ou raccrocher un appel de la même manière sur la plupart des appareils.
\\ \\
Pour cela, les terminaux de la ToIP utilisent le protocole SIP \textit{Protocole d'Initiation de Sessions}. Les terminaux communiquent avec un serveur dédié à la signalitique et la gestion des communication (Provider SIP) qui se chargera d'envoyer une trame pour effectuer une action.

\begin{figure}[H]
      \centering
      \includegraphics[width=\textwidth - \textwidth / 3]{ressources/r3rom16/01.png}
      \caption{Exemple simplifié de signalisation entre deux appareils utilisant le même Provider SIP.}
      \label{fig:echantillonnage}
\end{figure}

\noindent Nous avons ainsi analysé et étudié le trafic SIP généré pour différentes actions téléphoniques avec l'analyseur de trâmes Wireshark. Nous comprennons désormais les notions utilisées pour les communications d'entreprise basées sur de la ToIP.
    \renewcommand{\figurename}{}
\mychapter{R3.02 Réseaux d'Opérateur (22h30)}{cap:r302} 
\lhead{R3.02 Réseaux d'Opérateur (22h30)}
% tunnel gre et ipsec
% ospf et ospfv2avec priorisation
% routage inter-vlan avec détection des problèmes

\vspace*{0.2cm}%
      \large
      \href{}{\color{black}Enseignant\\M. Philippe Arnould}\\%
      \normalsize
\vspace*{0.5cm}%

Ce module consacré au réseau, complémenté par le module R3.07, nous ont fait étudier des technologies relatives aux réseaux d'opérateurs. Parmis eux dans ce module, le protocole de routage dynamique OSPFv2, le routage inter-vlan, les tunnels inter-réseaux GRE et les VPN IPsec.
\\ \\
L'enseignement se décomposait en séries de laboratoires Cisco que nous devions rendre à la fin des semaines.

\section{Le routage inter-vlan}

Le routage de VLANs \textit{Réseaux locaux virtuels} avait déjà été abordé dans le module R3.01. Il a été repris pour nous consolider avec la notion de routage inter-vlans on stick - utilisation d'un lien trunk pour transporter les VLANs.
\\ \\
Nous avons pu rechercher des indices de malfonctionnement d'une structure, en la réparant par la suite pour nous assurer que nous étions à l'aise avec le routage inter-vlan. Nous l'avons aussi déployé dans différent scénarios selon différents besoins.

\section{Routage dynamique avec OSPFv2}

Le protocole OSPF a été repris du module R3.01 et vu sous un autre angle. Toujours avec des laboratoires Cisco, nous avons étudié des parties spécifiques du protocole (messages "HELLO", intervale des messages...) et encore une fois dépanné plusieurs scénarios de malfonctionnement.

\section{Tunnels entre réseaux et Réseaux Privés Virtuels}

Nous avons abordés les VPN, ou \textit{Réseaux Privés Virtuels}, en étudiant le protocole IPsec. Celui-ci est une suite d'algorithmes et de protocoles permettant la création de tunnels chiffrés virtuels entre deux réseaux séparés sur Internet.
\\ \\
IPsec utilise notamment IKEv2 pour l'échange de clés entre les deux sites. Une fois les clés échangées, IPsec rencapsule les paquets IP à destination du site distant pour rajouter des entêtes comme ESP ou AH \textit{Entêtes d'Authentification} afin de garantir l'intégrité et la confidentialité des données.
\\ \\
Nous avons aussi vu les tunnels GRE \textit{Encapsulation de routage générique} qui permettent de nous simuler comme si nous étions dans le réseau distant à l'autre bout du tunnel. 
\\
Nous avons abordé les deux notions ensemble, et vu que les tunnels GRE pouvaient être encapsulés dans IPsec, pour intégrer le réseau local du site distant avec les avantages d'IPsec.
    \renewcommand{\figurename}{}
\mychapter{R305 Chaîne de transmission numérique (27h)}{cap:r305}
\lhead{R305 Chaîne de transmission numérique (27h)}

\vspace*{0.2cm}%
      \large
      \href{\@orientadorPagina}{\color{black}Enseignant\\Mr. Angel Abénia}\\%
      \normalsize
\vspace*{0.5cm}%

Le module R305 a été abordé pendant la première période à l'IUT, son examen s'est déroulé sur la deuxième période pendant une heure. Ce module avait comme objectif la meilleure caractérisation des supports de transmission en étudiant de manière plus approfondie la transmission des signaux, donc leur étude.

\section{Caractérisation d'un canal de transmission}

Une mise en matière à ce module pourrait être les informations suivantes.
\begin{itemize}
  \item Pour communiquer, les appareils ont besoin d'avoir des informations à échanger, sinon ils n'auraient pas eu besoin d'avoir à communiquer.
  \item Les informations sont véhiculées par un signal, électromagnétique, optique, électrique ou hertzien.
  \item Un support, ou canal de transmission, est utilisé pour transmettre les signaux. Celui-ci est adapté selon la nature du support physique et le type de signal à transporter.
\end{itemize}

\begin{figure}[h]
    \centering
    \includegraphics[width=1\linewidth]{imgs/support.png}
    \caption{Schéma représentatif d'un canal de transmission}
    \label{fig:canal}
\end{figure}

Un support de transmission admet une Bande passante \textbf{Bp}. Celle-ci est caractérisée par l'ensemble des fréquences que le canal permet de transporter. Ainsi que leur localisation dans l'espace des fréquences. Une bande passante admet donc une fréquence de début, une de fin, et une dernière centrale.
\\ \\
Bp de 100 Hz centrée sur 50 Hz. Soit Bp permet de transmettre toute fréquence dans l'intervale [0 Hz ; 100 Hz]
\\ \\
À leur dépassement à ses extrémités, le signal ne sera pas bien transmis; trop de dégradations seront appliquées pour les fréquences en dehors de la bande passante. Si nous pouvions avoir un canal de transmission parfait, avec aucune limite de bande passante et sans atténuation, alors le débit maximal théorique obtenable serait parfait aussi, outrepassant toute loi de la physique mais utile pour des calculs.
\\ \\
Ainsi, un canal de transmission avec une meilleure bande passante permet un meilleur débit binaire théorique : e.g. avec la fibre optique et une paire torsadée. Le signal "ne va pas plus vite", la vitesse de l'électricité dans un conducteur étant relativement proche de celle de la lumière dans du silice, mais la bande passante permise par le changement de support permet un meilleur débit, par la fibre moins d'atténuation, facilitant l'élargissement de la bande passante.
\\ \\
Dans la caractérisation de la chaîne de transmission du monde du numérique, nous avons rappelé le principe de liaison synchrone et asynchrone (synchronisation des horloges ou resynchronisation du front de décision à chaque information). Nous avons aussi revu le principe de débit binaire maximal théorique (brut), et celui dit "net" à l'utilisation - après toutes les informations nécessaires à la transmission des données ajoutées pour assurer sa transmission (dans les entêtes et queues des couches des protocoles).
\\ \\
Dans la caractérisation d'un signal, nous avons vu l'apparition des zones de décision pour différencier les états significatifs d'un signal. Pouvant se traduire par la plage de décision permise pour définir si l'on peut différencier un état significatif du signal (codant une information), pouvant laisser place à une zone d'incertitude. La place de décision doit toujours être inférieure au temps d'un état, sinon possibilité de confusion.

\begin{figure}[h]
    \centering
    \includegraphics[width=1\linewidth]{imgs/td.png}
    \caption{Schéma représentatif de la décision d'un état pour une sinusoïde simple, ou "comment définir l'état d'un signal"}
    \label{fig:td}
\end{figure}

Par ce schéma, on peut en ressortir que si une plage de décision est trop large comparée au moment d'un symbole (état significatif d'un signal) : on peut confondre un état pour un autre. Il s'agit de l'interférance inter-symbole, ici représenté par deux états électriques +A -A (A pouvant être négatif). L'intérêt est de bien configurer les seuils de décisions (encadrement des valeurs du signal - plage d'incertitude) et les temps de décisions (plage de décision d'un moment).
\\ \\
Un autre moyen plus simple de mettre en évidence l'interférance inter-symbole est de diagramme de l'oeil. Dans celui-ci sont défini tous les passages possibles des états.

\begin{figure}[h]
    \centering
    \includegraphics[width=1\linewidth]{imgs/seuils.png}
    \caption{Introduction au diagramme de l'oeil par identification des états d'un signal carré sans bruit (très fin)}
    \label{fig:seuils}
\end{figure}

Le temps où l'on peut définir si l'état est à 0 ou 1 ne peut être plus grand ou égale qu'à celui des symboles, sinon interférence inter-symbole. Le temps d'un symbole commence dès le changement du précédent état. Si l'instant de décision n'est pas correcte, deux choix pour atteindre les seuils si on ne peut pas les modifier : diminuer le débit pour faire rentrer le signal dans leur intervale, ou augmenter la bande passante - en changeant de canal de transmission.
%\\ \\
%Nous avons fait la différence entre le débit binaire brute théorique attégnable, et celui reçu après informations rajoutées aux données envoyées (destination dans un réseau, gestion de sessions, encapsulation de protocoles...) dans des entêtes et/ou des queues.
\\ \\
Nous pouvons changer le débit en conservant une bande passante correcte en jouant sur la valence du signal. En conservant la même fréquence, nous pouvons moduler le signal en amplitude ou en phase afin d'avoir davantage de symboles. La limite de cette pratique nous est donné par les travaux de Mr. Shanon que nous avons étudié, fixant que la valence se limite à ce que permet le rapport signal sur bruit, pour distinguer tous les états.
\\ \\
Nous avons aussi repris les travaux de Mr. Nyquist en intégrant ses critères pour définir la capacité d'un canal, soit sa bande passante maximale dans le domine théorique et physique. Nous avons notamment démontré la différence entre ces deux en travaux pratiques.
\\ \\
Si le débit binaire augmente, la bande passante aussi, sinon interférence entre les symboles. Pour contrer ceci soit augmenter la bande passante, soit réduire le débit, soit jouer sur les seuils et les instants de décision. D'autres états peuvent être introduits par modulation, à question que ceux-ci puissent être distingués, donc pas limités par le rapport entre le niveau de puissance du signal et celui du bruit.

\section{Fiabilisation d'une transmission}

La fiabilisation d'une transmission peut se caractériser par sa capacité à transmettre un message selon des circonstances données. Ainsi, des mécanismes de contrôle d'erreurs sont instaurés avec une demande de ré-émission par exemple. Le choix du type de transmission est aussi important, sa modulation. Dans cette partie nous avons abordé les codecs et nous avons étudié les types de modulation (manchester, NRZ...).
\\ \\
Certains types de modulations permettent une meilleur résistance au bruit, notamment ceux par phase vu diagramme de constellation. Chacun code le signal comme il le souhaite, le temps et les chercheurs sont par exemple passés du codage NRZ \textit{non-return-to-zero} à du PSK \textit{Phase-shift keying} ou du QAM \textit{quadrature amplitude modulation} toujours utilisés aujourd'hui. Leur largeur de spectre pour les mêmes informations envoyées changent aussi, pareil que pour son emplacement dans un spectre d'amplitude (lobe principal centré sur 0 Hz pour ceux qui ne modulent pas en fréquence...).

\section{Aboutissants du module}

Nous avons approfondi nos connaissance dans les supports de transmissions qui nous servent aujourd'hui. Nous pouvons les caractériser correctement pour montrer une infrastructure, nous pouvons les diagnostiquer dans nos domaines de compétences. Nous comprenons désormais comment circule un signal dans la "chaine du transmission du numérique", son histoire et son arrivée au monde moderne.
    \renewcommand{\figurename}{}
\mychapter{R3.06 Fibres optiques et propagation (19h30)}{cap:r306} 
\lhead{R3.06 Fibres optiques et propagation (19h30)}
    \renewcommand{\figurename}{}
\mychapter{R3.09 Programmation événementielle (15h)}{cap:r309} 
\lhead{R3.09 Programmation événementielle (15h)}

\vspace*{0.2cm}%
      \large
      \href{}{\color{black}Enseignant\\M. Manuel Munier}\\%
      \normalsize
\vspace*{0.5cm}%

Unique module de programmation en deuxième année, celui-ci proposait de prendre en main la programmation par intervention d'événements; la programmation événementielle. En outre, de concevoir une application capable de déclencher des événements à l'exécution de telle ou telle interraction (interrogeant des fonctions du code).
%\textit{"callback"})
\\ \\
L'objectif final de ce module était de créer une application avec un code commenté, un rapport décrivant le travail apporté te la logique de nos programmes, pour enfin fournir une courte présentation et une démonstration à l'issue.

\section{Conception d'une application de messagerie Python}

Le sujet de l'application était imposé : un explorateur MQTT. Nous devions utiliser l'interfaçage graphique de la librairie TKinter déjà utilisée en première année. Nous devions programmer les boutons de l'interface pour effectuer des appels de fonctions dans notre code; notre action étant l'événement dans ce cas. Il était aussi demandé de faire du multithreading, en soit la division de notre code en plusieurs parties légères, plutôt qu'un grand programme gérant le tout.
\\ \\
En backend \textit{derrière l'application, ce que l'utilisateur ne voit pas}, nous nous connections à un broker MQTT publique, auquel nous souscrivions à un topic unique. Suite à quoi, nous pouvions envoyer des messages à ce broker que toutes les personnes abonnées au même topic que nous pouvez recevoir (et inversement).
\\ \\
L'intérêt du frontend \textit{interfaçage utilisateur} était de pouvoir s'abonner à des topics, recevoir et envoyer des messages en appelant des fonctions backend.
\\ \\
\begin{figure}[H]
      \centering
      \includegraphics[width=\textwidth - \textwidth / 8]{ressources/r309/00.png}
      \caption{Utilisation de l'applicatif MQTT Explorer pour vérifier le fonctionnement d'un de mes scripts.}
      \label{fig:r309-00}
\end{figure}
    \renewcommand{\figurename}{}
\mychapter{R3.11 Anglais : le monde du travail (22h30)}{cap:r311} 
\lhead{R3.11 Anglais : le monde du travail (22h30)}
    \renewcommand{\figurename}{}
\mychapter{R3.14 Mathématiques: Analyse de Fourier (27h)}{cap:r314} 
\lhead{R3.14 Mathématiques: Analyse de Fourier (27h)}

    % Pós Textuais
    % \nocite{*}
    % \include{pos-textuais/referencias}

\end{document}
