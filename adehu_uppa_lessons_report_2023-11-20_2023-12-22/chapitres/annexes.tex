\mychapter{Annexes}{cap:annexes}
\lhead{Annexes}

% \begin{figure}[H]
%     \centering
%     \includegraphics[width=\textwidth - \textwidth / 5]{zone_chalandise_aditu.png}
%     %\includegraphics[scale=0.2]{zone_chalandise_aditu.png}
%     \figurename
%     \caption{Visualisation de la zone de chalandise d'ADITU, regroupée autour de ses datacenters à Bidart et à Dax}
%     \label{fig:zone_chalandise}
% \end{figure}

% \section{Cahier des charge Ticketing}

% % \begin{figure}[H]
% %     \centering
% %     \includegraphics[width=\textwidth - \textwidth / 20]{CDC-Ticketing.pdf}
% %     \figurename
% %     \caption{Fiche de poste de notre alternance}
% %     \label{fig:poste}
% % \end{figure}

% \textbf{Cahier du logiciel de Ticketing}

% \textbf{Infrastructure actuelle}

% L'historique

% Notre logiciel de ticketing actuelle est historique. En 2012 un projet
% de CRM est lancé et s'est soldé par le choix de la solution Vtiger. Pour
% répondre correctement au besoin d'ADITU, nous missionnons un développeur
% pour réaliser certains développements sur cette solution.

% Sur cette solution Vtiger, nous utilisions les modules suivants~:

% \begin{itemize}
% \item
%   Contact client et prospect
% \item
%   Devis \& Facturation
% \item
%   Ticketing
% \end{itemize}

% Avec le temps nous découvrons que les modifications ne sont pas
% compatibles avec les futures versions de Vtiger. Il en résulte que nous
% n'avons pas suivi les évolutions de version et notre Vtiger est resté
% figer en version 5.4.0 de 2012 alors qu'en 2023 nous sommes en V9.

% À ce jour, la partie commerciale (Devis \& Facturation) a été migrer sur
% le logiciel Cloud Axaunot.

% Il reste maintenant à migrer la partie ticketing.

% Avantage du Ticketing Vtiger

% Le seul bon point que l'on puisse retenir sur cette solution est d'avoir
% dissocié le frontend client et backend admin. La base de données est
% hébergée sur un serveur en local et un module web client était installé
% sur un serveur web. Ce genre de topologie n'a pas était retrouvé dans
% les logiciels similaires.

% Le frontend est accessible sur~:

% \url{https://support.aditu.fr/login.php}

% Serveur~: \textbf{85.31.144.198 -- ADT-MUTU-WEB3}

% Le backend est accessible sur~:

% \url{http://intranet.aditu.fr/}

% Serveur~: \textbf{85.31.147.204 - ADT-DEB-INTRANET}

% Les flux réseau entre ces deux serveurs sont restreints pour autoriser
% seulement l'API Vtiger à discuter avec le serveur backend. Après
% vérification, les règles n'ont plus l'air d'être présentes sur le
% firewall SN500.

% Interface du frontend

% Liste des tickets de l'utilisateur

% \includegraphics[width=6.3in,height=3.12222in]{image1.png}

% Création de tickets

% \includegraphics[width=6.3in,height=3.14722in]{image2.png}

% On peut constater que l'interface est plutôt minimaliste et
% vieillissante. Il manque certaines fonctions qui seront abordées plus
% bas.

% \textbf{Nouveau logiciel souhaité}

% Nous souhaitons migrer vers un nouveau logiciel de ticketing qui puisse
% intégrer les fonctionnalités ci-dessous~:

% Création d'incidents

% \begin{quote}
% La gestion des incidents permet de suivre et de résoudre les incidents
% signalés par les utilisateurs.
% \end{quote}

% Création de demandes de changement

% \begin{quote}
% Gestion des changements permet de planifier, suivre et gérer les
% modifications demandées par les utilisateurs.

% Exemple~: modification de ports sur le firewall, augmenter une boite aux
% lettres, etc.
% \end{quote}

% Création de tickets automatiquement

% \begin{quote}
% Cette fonctionnalité permet de planifier des interventions récurrentes
% sans les oublier. Exemple~: test de restauration
% \end{quote}

% Gestion des ressources

% \begin{quote}
% Pouvoir lié du matériel ou logiciel a un client et faire un suivi des
% modifications sur ce cette ressource.

% Exemple~: avoir le suivi des modifications comme celui qui est présent
% sur la page client du wiki.

% Intégrer du matériel lié à un fournisseur et gérer sa garantie. Exemple
% NAS ANANDA.

% Gérer les renouvellements~: des certificats

% Gérer les renouvellements~: des noms de domaines
% \end{quote}

% Inventaires

% \begin{quote}
% Lister l'ensemble des machines (connecteur OCS)
% \end{quote}

% Tableaux de bord et rapports

% Avoirs des stats et indicateurs sur ce qui nous prend le plus de temps
% dans le support.

% Ergonomique

% \begin{quote}
% Il faut que le logiciel soit intuitif et ergonomique. Que l'utilisateur
% ne soit pas rebuté par la complexité d'ouverture d'un ticket.
% \end{quote}

% Ouverture de ticket via email

% Fermeture de ticket automatique après un délai de non-réponse

% Personnalisation graphique

% \begin{quote}
% Nous souhaitons que l'application puisse se personnaliser aux couleurs
% de la société et d'y insérer le logo.
% \end{quote}

% Récupération des données ticketing vtiger

% \begin{quote}
% Seulement si cette récupération est facile. Ne pas perdre du temps sur
% cette récupération.
% \end{quote}

% Fonctions annexes

% Gestion des baies racks du datacenter
