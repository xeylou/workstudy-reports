\begin{resumo}
    Pour ma deuxième période en entreprise, je me suis vu attribuer la responsabilité de commencer l'étude d'une nouvelle solution de communication avec les clients. Dans cet exercice, je me suis accordé à attentivement écouter les attendus de mes collègues ainsi que des clients, ce qui m'a permis d'approfondir mes connaissances sur ce que je pensais connaître de la communication dans la prestation de services informatiques. Mon objectif fut par la suite d'agréger toutes ces informations afin d'avoir une idée globale de la solution que chacun désirait.
    \\ \\
    S'en est suivi une phase d'étude comparative et budgétaire des solutions disponibles pour connaître lesquelles serraient susceptibles de répondre au mieux à ces besoins. Une fois ces solutions choisies, une phase d'approche en laboratoire a été faite pour chacune d'entre elles afin de comprendre leur fonctionnement en conditions simulées; pour appuyer notre étude et ne retenir qu'une solution après comparaison.
    \\ \\
    Le premier grand pas fut par la suite de savoir intégrer les besoins de tous à cette solution. En déployant les fonctionnalités attendues, les suggestions apportées; en soit réarranger la solution pour accueillir notre support client et autre.
    
    \palavrasChave{recherche de solutions; autonomie; communication; compréhension des besoins; documentation; déploiement d'applicatif; rédaction; travaux de recherche}
\end{resumo}