\renewcommand{\figurename}{}
\mychapter{Étude d'une nouvelle solution de communication cliente}{cap:etude_d_une_nouvelle_solution_de_communication_cliente} 
\lhead{Étude d'une nouvelle solution de communication cliente}

Les solutions de ticketing \textit{solutions de support par tickets}~ "CRM" \textit{Gestion de la relation cliente} ou encore les "ITSM" \textit{Solutions de gestion de services informatiques} sont des éléments clés d'une équipe informatique pour gérer ses interactions avec ses utilisateurs. Cela s'applique aux services informatiques d'une entreprise pour les autres entités de l'entreprise, ou pour une société de prestation de services informatiques à ses clients comme c'est le cas pour ADITU.
\\ \\
Ces solutions permettent un suivi simple et pérenne des demandes et des incidents des utilisateurs. Elles facilitent le travail des équipes informatiques en permettant un suivi plus simple des demandes (cycle de résolution d'un ticket, assignation d'une demande, archivage...), de créer des routines de travail autour de cette gestion (réunions, bilan semestriel...) et d'améliorer l'interfaçage avec les utilisateurs (formulaires préremplis, foire aux questions...).

\section{Pourquoi un changement de solution}

ADITU possédait déjà une solution de communication avec ses clients par tickets. Les demandes et les incidents n'étaient pas différenciés, ils étaient gérés de la même manière. Cet outil possédait une interface vieillissante, simpliste et non agréable visuellement et à l'utilisation. L'équipe se plaignait également d'un manque de praticité de l'interface pour la gestion des tickets et d'une trop faible flexibilité dans son utilisation (impossible de faire du cas par cas...).
\\ \\
Pour remédier à ces problèmes, ADITU a voulu arborer une nouvelle solution de support répondant à ces problèmes en plus d'ajouter de nouvelles fonctionnalités non présentes auparavant; qui faciliteraient le quotidien des utilisateurs et des membres de l'équipe.

\section{Les attendus des partis}

Du passage de l'ancienne à la nouvelle solution de support, mes collègues et les clients avaient chacun leurs attendus. Les comprendre m'a été contingent au choix de la solution à et de la structure à monter.

\subsection{Les attendus mes collègues}

Une interface davantage à jour a été la première information qui m'ait été donnée quant au changement de notre interface. L'ancienne avait fait son temps, au delà de la non praticité, avoir une belle interface au quotidien, surtout pour un outil qu'on utilise grandement est un élément à ne pas négliger dans son quotidien professionnelle. 
\\ \\
Le choix d'une solution flexible était aussi demandé. Les demandes des clients pouvant être variées, il était essentiel de pouvoir faire du cas par cas et d'avoir une certaine flexibilité dans la gestion des demandes (mise en attente, ré-assignation, indice de résolution différent...).
\\ \\
Il est aussi possible d'intégrer l'interface avec d'autres outils comme un annuaire (Active Directory, LDAP...), et de pouvoir gérer plus facilement ses informations avec son CMDB \textit{Solution de gestionnaire d'inventaire} meilleur que son prédécesseur. Ainsi, celle-ci peut gérer les prospects, les clients actuels, et l'inventaire en rapport; informations y sont facilement réutilisables pour d'autres projets.

\subsection{Les attendus de nos clients}

Je pense que l'attendu de nos clients va au plus simple : quand ils veulent faire part d'une demande ou d'un incident rencontré, cela doit être au plus simple pour leur prendre le moins de temps dans leur travail. Je me suis efforcé d'intégrer ce besoin dans la nouvelle interface.
\\ \\
Une jolie interface donne envie à un client de vouloir réutiliser un outil, plutôt que de le délaisser voire le repousser à l'utilisation. Ainsi, un attendu commun était une interface plus moderne, en restant simpliste de leur point de vue.

\section{Les solutions étudiées}

Le cahier des charges stipulait les aspects obligatoires que devait arborer la prochaine solution de support en ligne. L'une d'entre elles était son coût financier 0. Ainsi, sur cette base de gratuité, j'ai recherché les solutions d'ITSM et de CRM gratuites répondant un maximum aux attendus cités plus tôt, ou qui s'en apparentait le plus. La solution devait être hébergeable chez ADITU.
\\ \\
J'ai préféré ne pas prendre de solutions trop juvéniles, pour avoir une communauté reconnue/expérimentée derrière et un bon support. J'ai aussi évité les solutions dites freenium (une partie du logiciel gratuite mais les intégrations payantes).
\\ \\
Deux solutions répondaient alors à ces valeurs après analyse : Combodo iTop et Teclib' GLPI.

\section{Une première approche en laboratoire}

Une fois les solutions survolées, je me suis laissé une semaine pour monter un environnement de simulation contrôlé avec chacune afin de les mettre à l'épreuve dans un cadre contrôlé. Je n'en ai pas fait une utilisation poussé, juste du renseignement sur leurs intégrations, et les comparer sur les attendus et les demandes du cahier des charges. 

\section{L'étude comparative}

Pour les comparer, j'ai décidé de noter les solutions sur les domaines abordés par le cahier des charges, les attendus de l'équipe et des clients. J'ai noté la réponse des solutions aux demandes de 0 à 3, 0 étant la solution ne répond pas au besoin demandé et 3 elle y répond au mieux.
\\ \\
Les scores de celles-ci peuvent augmenter ou diminuer si elles remplissent mal ou bien leur tâche, ou si leur implémentation est trop complexe ou brouillon à mettre en place ou à maintenir par exemple. À garder en mémoire que le score de l'une peut influencer sur celle de l'autre, elles sont ici comparées.
\\ \\
En voici le tableau comparatif.
\newpage
\begin{table}[h!]
% \begin{tabular}{|@{}l|l|l@{}|}
\begin{tabular}{|l|l|l|}
\hline
Réponse au cahier des charge par les solutions            & Combodo iTop & Teclib' GLPI \\ \hline
Ticketing                                                 & 3            & 3            \\
Devis et Facturations                                     & 3            & 0            \\
Front-end client et Back-end admin                        & 3            & 2            \\
Création d'incidents                                      & 2            & 2            \\
Demandes de changements par les utilisateurs              & 3            & 2            \\
Création de tickets automatique                           & 3            & 3            \\
Suivi des modifications des demandes                      & 3            & 3            \\
Connecteur OCS pour informations                          & 3            & 3            \\
Tableau de bord personnalisable                           & 3            & 2            \\
Fermeture automatique de tickets                          & 2            & 3            \\
Intégration de la charte graphique de Aditu               & 2            & 2            \\
Récupération des tickets de vTiger                        & 2            & 2            \\
Gestion des baies et racks du datacenter (CMDB)           & 3            & 3            \\
Ouverture de ticket par envoi de mail                     & 3            & 3            \\
Gestion des ressources (certificats, noms de domaine...)  & 3            & 3            \\
KB/FAQ                                                    & 3            & 2            \\
Envoi de mails sur l'information d'activité d'une demande & 3            & 3            \\
Ajout d'informations individuelles sur le portail client  & 3            & 1            \\
Gérer les lieux des clients                               & 3            & 3            \\ \hline
Score total (sur 57)                                      & 51           & 45           \\
\hline
\end{tabular}
\end{table}

\section{Premiers pas dans l'intégration}

Combodo iTop est donc la solution pour laquelle nous avons choisi d'opter. Après en avoir informé mon tuteur par plusieurs mises aux points, je me suis attelé à son installation. Celle-ci devra être faite dans un environnement de pré-production. La dernière marche à passer sera sa mise en production, dans un environnement de production.
\\ \\
J'ai donc commencé la rédaction d'une documentation pour l'installation et l'utilisation de Combodo iTop, utile lorsqu'une autre personne souhaite se pencher sur son fonctionnement en cas de problème ou pour apprendre à mieux l'utiliser. J'y ai notamment documenté le déploiement de sa machine virtuelle, sa configuration. J'ai raccourci les explications en montant un script d'installation complet (interactif, commenté avec gestion d'erreur).
\\ \\
Les intégrations venant par la suite, et voulant faire les choses proprement et ne pas me presser par le temps : j'ai décidé de reprendre le travail de configuration à mon retour pour la troisième période - préférant peaufiner mon travail existant, vu avec mon tuteur.