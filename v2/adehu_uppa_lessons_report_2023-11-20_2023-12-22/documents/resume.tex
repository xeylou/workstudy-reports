\begin{resumo}
    Pendant cette période de cinq semaines ont été abordés des modules d'enseignement influant sur l'ensemble des blocs de compétences en deuxième année. Le réseau a été abordé par la vision des réseaux opérateurs avec des tunnels d'accès privés, des protocoles de routage ou la gestion de la ségmentation des réseaux. La téléphonie sur IP a aussi été vu en montant des infrastructures téléphoniques en réseau permettant le transport de signalitique pour passer des appels et la voix (voix sur IP).
    \\ \\
    En télécommunications nous avons abordés un module conséquent sur la fibre optique en comprennant les principes de longeurs d'ondes, des fibres monomodes et multimodes et l'utilisation d'un réflectomètre. Nous avons aussi amélioré notre compréhension des liaisons synchrones et asynchrones.
    \\ \\
    En programmation et en mathématiques, nous avons entrepris la création d'une application à gestion d'événements et la compréhension mathématique des séries de Fourier.
    
    \palavrasChave{téléphonie sur IP; voix sur IP; codecs; framing; VAD; Cisco; opérateur; ospf; VPN; VLAN; liaisons synchrones; rapport signal/bruit; valence; horloges électroniques; fibres; réflectomètre; longeur d'onde; Python; interface graphique; grammaire; TOEIC; séries de Fourier; signaux périodiques}
\end{resumo}