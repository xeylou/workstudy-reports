\renewcommand{\figurename}{}
\mychapter{Assainissement des VLANs du data centre de Dax}{cap:vtp} 
\lhead{Assainissement des VLANs du data centre de Dax}

Le protocole VTP \textit{VLAN Trunk Protocol} permet la déclaration de VLANs \textit{Virtual LANs} sur plusieurs switchs en les renseignant qu'une seule fois sur un switch serveur. Les switchs clients apprennent du switch serveur les VLANs utilisés dans le réseau.
\\ \\
Ce protocole est propriétaire des équipements Cisco. Le data centre de Dax DATA3 incorpore le protocole VTP dans son réseau pour DATA3 et ses clients. Cependant, différentes personnes s'en sont servies pour des usages différents, celui-ci est désormais divisé en sections de réseaux (avec plusieurs switchs serveurs VTP).
\\ \\
L'objectif de cette mission est d'assainir l'implémentation du protocole VTP dans le data centre en laissant uniquement les switchs "coeur de réseaux" distribuer les VLANs. Ce sera aussi l'occasion de mettre à jour les versions du protocole VTP installés sur les équipements et de documenter l'intégration pour éviter d'à nouveau avoir différentes implémentations.

\section{Fonctionnement du protocole VTP}

VTP intervient dans un souci de répétition de la déclaration manuelle des VLANs sur les switchs d'un réseau. Dans un réseau de data centre, cela devient vite impossible de les déclarer ainsi sur chaque switch individuellement à grande échelle.
\\ \\
Le protocole fonctionne sur le principe client/serveur. Le serveur seul peut déclarer des VLANs sur un réseau, lui-même composé de switchs clients en écoute du serveur pour assimiler de nouveaux VLANs. Le serveur et ses clients VTP doivent être dans le même domaine VTP. Un dernier mode "transparent" est adressable, permettant de laisser circuler les trâmes VTP sur le VLAN Default sans les assimiler. 
\\ \\
Une des fonctionnalités avancées du protocole VTP est son \textit{VTP Pruning} qui permet d'empêcher la diffusion de requêtes broadcast sur une partie du réseau si le VLAN n'y est pas utilisé (utile pour mitiger le traffic inutile).
\\ \\
Plusieurs versions de ce protocole existent et son inter-opératables : VTP-1, VTP-2 et VTP-3. La version 2 apportant du support pour l'architecture \textit{Token ring}, des trâmes de vérification de configuration (pas juste d'assimilation) et quelques modifications dans son fonctionnement.
\\ \\
La version 3 apporte une configuration avec authentification : n'importe qui ne pourra pas joindre un domaine VTP sans y avoir été invité, il suffisait jusqu'à présent de regarder les trâmes passées et de se joindre au domaine pour récupérer les VLANs et communiquer avec les machines de ceux-ci. Elle permet aussi d'utiliser l'intégralité du nombre de VLANs de la norme 802.1q (4094) comparé à 1006 pour les versions précédentes. 
\\ \\
À noter que les liens entre les switchs utilisant VTP doivent être en trunk, libre d'autoriser quels VLAN à circuler (le Default impérativement, 1 nativement, pour la circulation des messages VTP).

\section{L'infrastructure VTP actuelle}

L'infrastructure VTP était éclatée en deux domaines : un domaine pour le coeur de réseau et un autre pour le réseau de distribution/d'accès.
\\ \\
Dans ces deux réseaux, les serveurs VTP étaient empilés : un serveur VTP pouvait informer ses clients rattachés, mais pas ceux derrière le serveur VTP en dessous de lui (le tout sur le même domaine toujours).
\\ \\
L'infrastructure devait se débarrasser de l'empilement de serveurs VTP pour n'en permettre qu'à un seul uniquement de faire des annonces aux autres switchs et de pouvoir en créer.

\subsection{Relevé et changement des syntaxes des VLANs}

L'un des objectifs était donc d'unifier les deux domaines VTP présents en un seul.
\\ \\
Le problème étant qu'ils possédaient parfois les mêmes VLANs, avec ou sans même dénomination (e.g. "VLAN-0001" pour le coeur de réseau et "nom-du-client" pour le réseau d'accès).
\\ \\
J'ai donc répertorié les VLANs utilisés dans les deux domaines et en ai fait un tableur comparatif à montrer à Guillaume mon tuteur et Valentin mon superviseur pour cette tâche; afin qu'ils choisissent la dénomination à conserver pour les VLANs.  

\section{Essais en laboratoire}

Les manipulations sur le réseau d'un data centre doivent être validés par des essais au préalable pour ne pas engendrer une discontinuité de services chez nos clients. En adoptant ce raisonnement, j'ai monté un réseau virtualisé d'équipements Cisco afin de simuler l'intervention et constater d'éventuels problèmes réfléchis ou imprévus sur la structure avant qu'ils ne le soient sur le data centre.
\\ \\
À l'aide des outils de simulation Cisco Packet Tracer et GNS3, j'ai simulé une partie du réseau du data centre et essayé mes manipulations. Je les ai simulées avec des équipements qui s'apparentaient le plus aux nôtres en respectant leur version de Cisco IOS utilisée (un des systèmes d'exploitation d'équipements Cisco).
\\ \\
Dans mes manipulations, j'ai répliqué les configurations de nos équipements physiques pour au mieux simuler les éventuelles pannes suites à mes manipulations et de m'assurer de l'implantation de la nouvelle structure avec les anciennes configurations des équipements.
\\ \\
J'ai documenté mon approche et mes manipulations pour les reproduire à l'identique et pour qu'elles puissent être compréhensibles rapidement par quelqu'un qui souhaiterait reprendre le projet.

\section{Plan de modification}

Une fois un plan de nomenclature validé, avec une démonstration faite et approbation reçu par la simulation. Je pourrai commencer à assainir le plan d'adressage des VLANs du data centre de Dax à mon retour.
\\ \\
Je n'oublierai pas de documenter mes actions et d'expliquer la nouvelle infrastructure textuellement et graphiquement. Je conserverai toutes mes démarches (fichiers temporaires, actifs...) pour que n'importe qui puisse revenir sur le projet et comprendre son fonctionnement, les tenants et les aboutissants de la structure.
