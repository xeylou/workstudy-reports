\begin{resumo}
J'ai pour cette quatrième période effectué la mise en production de la nouvelle solution de support client. Je me suis aussi vu attribuer l'assainissement d'une partie du réseau du data centre de Dax, le tout en continuant mes travaux d'études pour une nouvelle solution de supervision.
\\ \\
Cette quatrième période annonce l'aboutissement de mon travail sur la nouvelle solution de support client. La deuxième m'ayant permis d'en apprendre davantage sur le support d'une ESN et d'en choisir la solution qui correspondait au mieux à nos attentes et nos besoins. La troisième à préparer professionnellement celle-ci en y adaptant l'ensemble des attendus des clients et de mes collègues. Quant à cette quatrième période, elle marque son déploiement et l'utilisation par les clients, ainsi que le maintient de la solution dans un contexte de production.
\\ \\
Mon premier travail en réseau fut d'assainir l'infrastructure des VLANs du data centre de Dax. Son application étant différente des réseaux domestiques, je me suis formé à utiliser le protocole de gestion de VLANs VTP. J'ai été en charge de comprendre le fonctionnement actif de la structure, de proposer un plan d'assainissement, de l'implémenter une première fois en environnement virtualisé et de prochainement appliquer ces changements au data centre.
\\ \\
J'ai en parallèle continué mes travaux sur l'étude d'une nouvelle solution de supervision, en mettant à l'essai la solution retenue de l'étude comparative : Zabbix. J'y ai appliqué des principes fondamentaux et avancés de la supervision dans un environnement contrôlé, puis recherché à comprendre son fonctionnement et à découvrir des pistes d'amélioration pour notre infrastructure actuelle.
    
    
    \palavrasChave{autonomie; étude de solutions en activité; mise en production d'applicatifs; compréhension et virtualisation de réseaux avancés; documentation; rédaction; travaux de recherche; suivi de l'activité d'une solution}
\end{resumo}
