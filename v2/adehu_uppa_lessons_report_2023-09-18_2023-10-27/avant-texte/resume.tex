\begin{resumo}
    Beaucoup d'enseignements ont été dispensés pendant ces six semaines à l'IUT. Tous les modules de compétences ont été abordés. Pour le module réseau, nos avons abordé les réseaux de campus avec l'utilisation de protocoles avancés pour la gestion d'un parc réseau conséquent. Nous avons aussi étudié et installé des réseaux d'accès opérateurs à technologie xDSL. Tout un module était dédié à la téléphonie sur réseau IP et les apports de cette solution.
    \\ \\
    En télécommunications, nous avons caractérisé d'une manière plus avancée qu'en première année les supports de transmission et certains types de communications - liaison synchrone et asynchrone. Nous l'avons lié aux réseaux à technologie xDSL : ses enseignements étaient complémentaires.
    \\ \\
    Nous avons abordé en parallèle un module de programmation orientée objet. Ainsi que les systèmes de bases de données en profondeur côté numérique. Nous avons aussi monté des services pour étudier en application les services de résolution de noms et les services d'envoi \& réception de mails, en plus des annuaires informatiques.
    
    \palavrasChave{réseaux; systèmes d'informations; sécurité; ToIP; VoIP; codecs, Cisco; SSH; protocoles de routage; DNS; serveurs mails; annuaires; tranmission; support de transmission; étude de l'information; réseaux d'accès; ADSL; bases de données; MongoDB; NoSQL; communication; rédaction;}
\end{resumo}