\begin{resumo}
    Cette troisième période me présentait deux axes de développement : la préparation de la nouvelle interface de communication client pour un déploiement prévu ma prochaine période, et le commencement de l'étude pour choisir une nouvelle solution de supervision de nos équipements \& de nos actifs, en data centres et chez nos clients. 
    %\\ \\
    %Comparé aux périodes précédentes, j'ai pu travailler ces exercices conjointement pour mieux organiser mon temps de travail : quand l'un était en attente, j'avançais sur l'autre...
    \\ \\
    La préparation de la nouvelle interface de support comprenait un nouveau déploiement sain et documenté, sa modification pour un lieu en production (exposé sur Internet, qui plus est à des clients), la configuration de l'outil et de nombreux tests de fonctionnement.
    \\ \\
    L'étude pour une nouvelle solution de supervision s'est composée de documentation pour comprendre les nouveaux et anciens principes de supervision, des astuces, méthodes et autres. Elle s'aggrémentait d'une écoute de nos besoins, d'une prise de connaissance de l'ancienne solution et des contraintes que nous avions.
    \\ \\
    Suite à quoi, j'ai monté un un environnement d'essais dans lequel j'ai pu étudier la réaction des actifs et mes connaissances apprises. Cela me permettait aussi d'en apprendre davantage sur les solutions de supervision grand public ou non pour comprendre vers laquelle se diriger pour nos besoins, comment et d'enrichir mes connaissances sur le domaine.
    
    
    \palavrasChave{recherche de solutions; autonomie; compréhension des besoins; documentation; déploiement d'applicatif; rédaction; travaux de recherche}
\end{resumo}