\renewcommand{\figurename}{}
\mychapter{Premiers pas vers Aditu}{cap:premier_pas_vers_aditu} 
\lhead{Premiers pas vers Aditu}

Cette première partie aborde la présentation approfondie de l'entreprise ADITU, en y présentant son équipe, son secteur d'activité et les services en réponse; sa clientèle et enfin son périmètre d'attraction commerciale. Une deuxième sous-section est dédiée à mon intégration dans l'entreprise, traitant de mes conditions de travail et d'accueil, pour arriver sur la présentation d'une journée type de mes activités chez ADITU.

\section{Présentation d'Aditu}

ADITU est le nom de l'entreprise dans laquelle j'évolue dans mon alternance. Signifiant "écouter" ou "expert" en basque, elle se définit comme une société de prestation de services informatiques, ou ESN \textit{Entreprise de services du numérique}, à l'écoute de ses clients.
\\ \\
Elle fut fondée en 2004 en tant que première Délégation de Service Public en France dans le domaine des services aux entreprises. Sa création ayant été initiée par la Communauté d’Agglomération Côte Basque-Adour (Bayonne, Anglet, Biarritz, Boucau et Bidart). Son implémentation se fut à la Technopole Izarbel, à Bidart.
\\ \\
Toute son équipe, y compris son Dirigeant  Mr. Éric Pierre-Sala mon responsable, et son Directeur Technique Mr. Guillaume Devesa mon tuteur, m'ont accueilli dans leurs locaux.

\subsection{Secteur d'activité}

Le secteur d'activité d'ADITU se caractérise par sa réponse aux besoins d'aider les entreprises du secteur local à se développer plus rapidement et efficacement en les déchargeant de l'élaboration ou le maintient de leur infrastructure informatique. ADITU tend à être l'intermédiaire simple des professionnels locaux au monde du numérique.
\\ \\
ADITU participe au développement territorial en prenant en charge le déploiement et le maintient des infrastructures informatiques des acteurs locaux. Elle leur permet de pouvoir se concentrer sur leur activité en leur déchargeant de l'utilisation d'une infrastructure faillible ou incorrectement maintenue (complications au quotidien, perte de données, complexité d'utilisation, problème d'expansibilité multi-sites, failles de sécurités...).  
\\ \\
Le secteur d'activité est solide, assurant aux entreprises leur bon fonctionnement sans qu'elles aient besoin de former ou d'embaucher une personne technique à temps plein pour son infrastructure. L'équipe d'ADITU travail à la performance et à la refocalisation des entreprises sur leurs activités principales.

\subsection{Services proposés}

Pour répondre aux besoins du secteur d'activité, ADITU propose des services de conseils, d'installation et de maintient d'infrastructures informatiques à la demande pour ses clients. Chaque client peut choisir la souscription à un ou plusieurs services selon leurs besoins et le support souhaité.
\\ \\
Pour assurer la continuité des services proposés, ADITU possède deux datacenters \textit{centres d'hébergement et de gestion de données}, un premier à Bidart et un deuxième à Dax. Ces datacenters servent à \textbf{l'hébergement} de machines dédiées aux clients ou à l'hébergement de leurs services (sites WEB, service de messagerie, stockage et partage de fichiers, sauvegardes externalisées de postes ou de fichiers, applications en tout genre).
\\ \\
ADITU propose aussi la supervision du fonctionnement des services de ses clients, hébergés dans les datacenters ou sur leurs sites. Le service \textbf{d'infogérance} permet au client d'être informé en tant réel de la disponibilité de ses services pour ses collaborateurs, et d'être aussi rassuré de la remise en fonctionnement et du suivi de leurs services. L'infogérance englobe le support téléphonique et informatique, avec l'aide de résolution de problèmes au quotidien.
\\ \\
Pour répondre à la criticité des services de certains clients, l'infogérance peut être accompagnée d'une \textbf{astreinte} afin de garantir l'intervention sur incident dans les 45 minutes suivant la remonté de problème. Ce service est proposé avec un numéro d'appel, 365 jours par an, 24h sur 24, sans interruption. Ce service est d'une grande force pour les groupes voulant prouver leur professionalisme à leurs clients, par la haute disponibilité de leurs services et une réponse sur incident sûre.
\\ \\
Des journées en \textbf{régie} sont proposés, permettant l'envoi d'un technicien dédié au support et au contrôle du bon fonctionnement de l'infrastructure cliente sur l'ensemble d'une journée. Le technicien est alors solicité pour les problèmes mineurs, des formations sur équipements, de l'installation de nouveau matériel ou de la relation à client. Ce type de journées est intéressant pour de grands sites, permettant la vérification d'une utilisation propre de l'infrastructure installée, pour faciliter l'activité principale du client.

%Pour simplifier l'accessibilité mono-site ou multi-site

%pour répondre aux besoins du secteur d'activité... ; hébergement, infogérance, messagerie, sauvegarde externalisée, astreinte, régie

\subsection{Zone de chalandise}

ADITU a été fondé en soutient aux entreprises du Pays Basque, des Landes et du Béarn. Sa zone d'attraction s'étend dans ces régions. Cette zone de chalandise reste locale mais très éclatée sur la côte Basque. Le Directeur d'ADITU, Mr. Pierre-Sala Éric, désir que l'entreprise garde cette zone de chalandise pour conserver la proximité avec ses clients et ses acteurs, afin de conserver une relation humaine avec eux.
\\ \\
ADITU n'est pas la seule entreprise dans son secteur d'activité à s'intéresser à cette zone de chalandise. Celle-ci possède de nombreux concurrents, de même ou plus grande envergure. La zone de chalandise se retrouve ainsi divisée entre les imposants groupes proposant des services de plus grande échelle, et les groupes de taille humaine comme ADITU ayant de la proximité avec ses clients et étant appréciés pour leur localité.

% Clients locaux (pays basques, landes); 

\subsection{Clientèle ciblée}

La clientèle ciblée par la zone de chalandise est variée, regroupant entreprises et organisations ayant besoin d'externaliser la gestion de leur infrastructure numérique. La plupart des clients d'ADITU sont des clients "historiques" avec plus de dix années d'ancienneté.
\\ \\
Les clients d'ADITU sont des clients souhaitant de la proximité et de l'humain pour leur informatique, pouvant parler avec une personne. Ces personnes veulent de la proximité, étant généralement attachés à leur région - de par leur activité ou leur clientèle, et sont demandeurs d'acteurs locaux.
\\ \\
La différence de taille parmis les groupes des clients reste considérable : pouvant aller d'un grand groupe souhaitant s'installer dans le pays basque à une TPE \textit{Très Petite Entreprise} dans le besoin de quelques services.

% clients "historiques"; ceux qui ne veulent pas payer une fortune dans office365; ceux qui veulent avoir de la proximité; les gens attachés à leur secteur et demandeurs d'acteurs locaux

\section{Encadrement dans l'entreprise}

Cette deuxième sous-section est consacrée à mon encadrement dans l'entreprise. Celle-ci y aborde mes conditions générales de travail chez ADITU (temps de travail, organisation, relation à clients et collègues, encadrement pour l'alternance...) ainsi que mon environnement quotidien de travail (type de bureau, matériel, déplacement...). Je finis par présenter une visualisation d'une journée type de travail. 

\subsection{Conditions de travail}

Spécifié dans mon contrat d'alternance, mon lieu de travail se situe dans les bureaux d'ADITU à Bidart, Pavillon Izarbel. J'y travaulle du lundi au vendredi de 9h à 12h, puis de 14h 18h. J'y retrouve mon responsable Éric, mon tuteur Guillaume ainsi que mes collègues ces mêmes jours.
\\ \\
L'organisation de la semaine prend place le lundi matin aux environs de 9h30, par une réunion technique et une autre d'exploitation. Ces réunions ont pour objectifs respectifs de définir l'état d'avancement des projets de chacun et la planification de leurs tâches pour la semaine. Les alternants participent à ces réunions au même titre que les autres membres de l'équipe.
\\ \\
J'ai souvent eu l'occasion de discuter avec mon tuteur Guillaume ou mon responsable Éric, en essayant le plus possible de ne pas empiéter sur leurs temps de travail respectifs. Je suis convié une fois par période d'entreprise à faire un point avec eux sur ma situation personnelle, professionnelle et scolaire. Nous parlons fréquemment entre collègues du même service. Je n'ai eu que rarement l'occasion de converger avec des clients (voir Première activités).
\\ \\
J'ai effectué une visite médicale du travail la première période en entreprise le 12 septembre 2023 à 11:30.

\subsection{Environnement de travail}

L'environnement de travail est propre et sécuritaire. Mon activité principale respose sur l'utilisation d'un poste de travail fixe, faisant partie du NOC \textit{Network Operations Center} d'ADITU. L'agencement du NOC forme un open space collaboratif. Certaines tâches m'ont demandé d'aller dans le datacenter d'ADITU pour des manipulations, sous surveillance et explications les premières fois.
\\ \\
Je n'ai pas besoin de me déplacer dans mon travail, je ne fais pas de clientèle commerciale ou technique, ni de manipulations sur le datacenter de Dax. J'ai l'occasion de questionner les membres de l'équipe sur des spécificités, de l'aide ou des conseils (toujours en essayant de leur emprunter la période de temps la plus courte pour ce qui est demandé).

\subsection{Journée type}

Avec la réunion technique et celle d'exploitation le lundi, mes journées se déroulent dans les bureaux d'ADITU, sur mon poste ou dans son datacenter. J'y effectue mes tâches mises au point la veille, entouré des autres personnes de l'équipe.
\\ \\
Le travail qui m'est demandé est souvent encadré par un cahier des charges à mon retour de période scolaire. J'y découvre mes activités pour la période d'entreprise, faisant des mises aux points régulières les lundi matin.