\documentclass[tcc]{ic}

\hypersetup{
    colorlinks = {true},
    linktocpage = {false},
    plainpages = {false},
    linkcolor = {Blue},
    citecolor = {Blue},
    urlcolor = {Red},
    unicode = {true},
    pdftitle = {Alexis Rapport d'apprentissage en entreprise ADITU du 01-09-2023 au 13-09-2023}, 
    pdfauthor = {Alexis Déhu},
    pdfsubject = {Rapport de mes activités chez ADITU pour la période du 01-09-2023 au 13-09-2023},
    pdfkeywords = {activités, alternance, iut, but, rt, aditu, uppa, pau, bidart}
}

\usepackage{algorithm}
\usepackage{algpseudocode}

\makeatletter
\renewcommand{\ALG@name}{Algoritmo}
\renewcommand{\listalgorithmname}{List of \ALG@name s}
\makeatother

\newcolumntype{Y}{>{\centering\arraybackslash}X}

\begin{document}

    % Inclui o preâmbulo do documento (Informações da capa e contracapa)
    \titulo{Rapport d'apprentissage en entreprise}

\autor{Alexis Déhu}{a.dehu@aditu.fr}{}


\orientador{M. Éric Pierre-Sala}{}{}{}{}
%\orientador{Mr. Devesa Guillaume}{}{}{}{}
\examinador{M. Guillaume Devesa}{}{}{}{}

% \examinadorDois{Nome}{}{Instituto de Computação}{Universidade Federal de Alagoas}{Prof. Me.}

\dataMesAno{Septembre}{2023}{}

    \selectlanguage{english}
    
    % Capa, contracapa e avaliadores
    \capa

    % \aprovacao
    
    % Agradecimentos
    % \begin{agradecimentos}

Obrigado

\end{agradecimentos}
    
    % Resumo e Abstract
    \begin{resumo}
    Pour ma deuxième période en entreprise, je me suis vu attribuer la responsabilité de commencer l'étude d'une nouvelle solution de communication avec les clients. Dans cet exercice, je me suis accordé à attentivement écouter les attendus de mes collègues ainsi que des clients, ce qui m'a permis d'approfondir mes connaissances sur ce que je pensais connaître de la communication dans la prestation de services informatiques. Mon objectif fut par la suite d'agréger toutes ces informations afin d'avoir une idée globale de la solution que chacun désirait.
    \\ \\
    S'en est suivi une phase d'étude comparative et budgétaire des solutions disponibles pour connaître lesquelles serraient susceptibles de répondre au mieux à ces besoins. Une fois ces solutions choisies, une phase d'approche en laboratoire a été faite pour chacune d'entre elles afin de comprendre leur fonctionnement en conditions simulées; pour appuyer notre étude et ne retenir qu'une solution après comparaison.
    \\ \\
    Le premier grand pas fut par la suite de savoir intégrer les besoins de tous à cette solution. En déployant les fonctionnalités attendues, les suggestions apportées; en soit réarranger la solution pour accueillir notre support client et autre.
    
    \palavrasChave{recherche de solutions; autonomie; communication; compréhension des besoins; documentation; déploiement d'applicatif; rédaction; travaux de recherche}
\end{resumo}
    % \begin{abstract}
%     Lorem ipsum dolor sit amet, consectetur adipiscing elit. Duis elit tellus, vehicula in justo eget, fermentum aliquet nisi. In id quam mauris. Sed id vehicula libero. Quisque id tortor placerat, consequat ex sed, placerat ante. Ut dui nunc, placerat volutpat ipsum sit amet, dictum pellentesque lacus. Donec leo sem, dictum quis lacus at, ultricies sagittis elit. Mauris sit amet tortor efficitur, luctus justo sit amet, tincidunt tellus. Donec vulputate non risus at tincidunt. Sed accumsan at erat in aliquet. Sed consequat gravida bibendum. Sed fermentum metus sed ex lacinia mattis. Phasellus vel enim nisl. Proin tortor dui, luctus in erat a, dapibus convallis turpis. Maecenas vitae vulputate neque.

%     Etiam ac ante a lorem consectetur varius. Donec vitae dui porttitor, efficitur ante sed, aliquet lectus. Etiam aliquet mattis sagittis. Integer accumsan, est nec vehicula suscipit, nibh nisl varius ante, at convallis urna est dapibus massa. Orci varius natoque penatibus et magnis dis parturient montes, nascetur ridiculus mus. Praesent tempus dolor sit amet metus eleifend porta. Vestibulum eget viverra nulla. Mauris ac condimentum augue, quis molestie tortor. Duis ac condimentum tortor, sed ullamcorper nunc. Vivamus et suscipit arcu. Duis eget rutrum est, sed pellentesque leo. Nulla lorem lacus, faucibus vitae lectus eu, porttitor efficitur ante. Sed risus tortor, venenatis quis convallis non, blandit in orci. Morbi et neque hendrerit, consectetur magna vitae, consectetur dolor. Nam nec iaculis urna, vitae tincidunt eros. Ut pretium neque convallis turpis rutrum, nec dapibus est faucibus.
    
%     Quisque id laoreet ligula. In hac habitasse platea dictumst. Fusce scelerisque, nunc non lacinia maximus, lorem metus rutrum lacus, ac maximus dolor tellus at velit. Sed diam leo, interdum ut sapien at, finibus aliquam diam. Praesent vel erat id enim scelerisque fringilla. Aliquam cursus risus vulputate ex interdum, sit amet pretium augue semper. Curabitur eget risus eget nulla placerat ornare. Nam quis ornare ipsum. Duis id feugiat lectus. Phasellus vehicula leo id consequat porta. Aliquam ut massa malesuada, ultricies ipsum nec, euismod eros. In lacinia aliquet leo non mattis. Etiam semper neque risus, a condimentum diam euismod eget. Suspendisse vulputate viverra mauris, ac sollicitudin tellus placerat non. Nullam felis metus, congue sed neque ac, iaculis sollicitudin nisl.
    
%     \keywords{graphics processing; medical images; computer vision; deep learning; data augmentation;}
% \end{abstract}
    
    % Sumário
    \renewcommand*\contentsname{Table des matières}
    \tableofcontents
    \thispagestyle{empty}
    
    % Início dos capítulos
    \inicio
    
    \renewcommand{\figurename}{}
\mychapter{Étude d'une nouvelle solution de communication cliente}{cap:etude_d_une_nouvelle_solution_de_communication_cliente} 
\lhead{Étude d'une nouvelle solution de communication cliente}

Les solutions de ticketing \textit{solutions de support par tickets}~ "CRM" \textit{Gestion de la relation cliente} ou encore les "ITSM" \textit{Solutions de gestion de services informatiques} sont des éléments clés d'une équipe informatique pour gérer ses interactions avec ses utilisateurs. Cela s'applique aux services informatiques d'une entreprise pour les autres entités de l'entreprise, ou pour une société de prestation de services informatiques à ses clients comme c'est le cas pour ADITU.
\\ \\
Ces solutions permettent un suivi simple et pérenne des demandes et des incidents des utilisateurs. Elles facilitent le travail des équipes informatiques en permettant un suivi plus simple des demandes (cycle de résolution d'un ticket, assignation d'une demande, archivage...), de créer des routines de travail autour de cette gestion (réunions, bilan semestriel...) et d'améliorer l'interfaçage avec les utilisateurs (formulaires préremplis, foire aux questions...).

\section{Pourquoi un changement de solution}

ADITU possédait déjà une solution de communication avec ses clients par tickets. Les demandes et les incidents n'étaient pas différenciés, ils étaient gérés de la même manière. Cet outil possédait une interface vieillissante, simpliste et non agréable visuellement et à l'utilisation. L'équipe se plaignait également d'un manque de praticité de l'interface pour la gestion des tickets et d'une trop faible flexibilité dans son utilisation (impossible de faire du cas par cas...).
\\ \\
Pour remédier à ces problèmes, ADITU a voulu arborer une nouvelle solution de support répondant à ces problèmes en plus d'ajouter de nouvelles fonctionnalités non présentes auparavant; qui faciliteraient le quotidien des utilisateurs et des membres de l'équipe.

\section{Les attendus des partis}

Du passage de l'ancienne à la nouvelle solution de support, mes collègues et les clients avaient chacun leurs attendus. Les comprendre m'a été contingent au choix de la solution à et de la structure à monter.

\subsection{Les attendus mes collègues}

Une interface davantage à jour a été la première information qui m'ait été donnée quant au changement de notre interface. L'ancienne avait fait son temps, au delà de la non praticité, avoir une belle interface au quotidien, surtout pour un outil qu'on utilise grandement est un élément à ne pas négliger dans son quotidien professionnelle. 
\\ \\
Le choix d'une solution flexible était aussi demandé. Les demandes des clients pouvant être variées, il était essentiel de pouvoir faire du cas par cas et d'avoir une certaine flexibilité dans la gestion des demandes (mise en attente, ré-assignation, indice de résolution différent...).
\\ \\
Il est aussi possible d'intégrer l'interface avec d'autres outils comme un annuaire (Active Directory, LDAP...), et de pouvoir gérer plus facilement ses informations avec son CMDB \textit{Solution de gestionnaire d'inventaire} meilleur que son prédécesseur. Ainsi, celle-ci peut gérer les prospects, les clients actuels, et l'inventaire en rapport; informations y sont facilement réutilisables pour d'autres projets.

\subsection{Les attendus de nos clients}

Je pense que l'attendu de nos clients va au plus simple : quand ils veulent faire part d'une demande ou d'un incident rencontré, cela doit être au plus simple pour leur prendre le moins de temps dans leur travail. Je me suis efforcé d'intégrer ce besoin dans la nouvelle interface.
\\ \\
Une jolie interface donne envie à un client de vouloir réutiliser un outil, plutôt que de le délaisser voire le repousser à l'utilisation. Ainsi, un attendu commun était une interface plus moderne, en restant simpliste de leur point de vue.

\section{Les solutions étudiées}

Le cahier des charges stipulait les aspects obligatoires que devait arborer la prochaine solution de support en ligne. L'une d'entre elles était son coût financier 0. Ainsi, sur cette base de gratuité, j'ai recherché les solutions d'ITSM et de CRM gratuites répondant un maximum aux attendus cités plus tôt, ou qui s'en apparentait le plus. La solution devait être hébergeable chez ADITU.
\\ \\
J'ai préféré ne pas prendre de solutions trop juvéniles, pour avoir une communauté reconnue/expérimentée derrière et un bon support. J'ai aussi évité les solutions dites freenium (une partie du logiciel gratuite mais les intégrations payantes).
\\ \\
Deux solutions répondaient alors à ces valeurs après analyse : Combodo iTop et Teclib' GLPI.

\section{Une première approche en laboratoire}

Une fois les solutions survolées, je me suis laissé une semaine pour monter un environnement de simulation contrôlé avec chacune afin de les mettre à l'épreuve dans un cadre contrôlé. Je n'en ai pas fait une utilisation poussé, juste du renseignement sur leurs intégrations, et les comparer sur les attendus et les demandes du cahier des charges. 

\section{L'étude comparative}

Pour les comparer, j'ai décidé de noter les solutions sur les domaines abordés par le cahier des charges, les attendus de l'équipe et des clients. J'ai noté la réponse des solutions aux demandes de 0 à 3, 0 étant la solution ne répond pas au besoin demandé et 3 elle y répond au mieux.
\\ \\
Les scores de celles-ci peuvent augmenter ou diminuer si elles remplissent mal ou bien leur tâche, ou si leur implémentation est trop complexe ou brouillon à mettre en place ou à maintenir par exemple. À garder en mémoire que le score de l'une peut influencer sur celle de l'autre, elles sont ici comparées.
\\ \\
En voici le tableau comparatif.
\newpage
\begin{table}[h!]
% \begin{tabular}{|@{}l|l|l@{}|}
\begin{tabular}{|l|l|l|}
\hline
Réponse au cahier des charge par les solutions            & Combodo iTop & Teclib' GLPI \\ \hline
Ticketing                                                 & 3            & 3            \\
Devis et Facturations                                     & 3            & 0            \\
Front-end client et Back-end admin                        & 3            & 2            \\
Création d'incidents                                      & 2            & 2            \\
Demandes de changements par les utilisateurs              & 3            & 2            \\
Création de tickets automatique                           & 3            & 3            \\
Suivi des modifications des demandes                      & 3            & 3            \\
Connecteur OCS pour informations                          & 3            & 3            \\
Tableau de bord personnalisable                           & 3            & 2            \\
Fermeture automatique de tickets                          & 2            & 3            \\
Intégration de la charte graphique de Aditu               & 2            & 2            \\
Récupération des tickets de vTiger                        & 2            & 2            \\
Gestion des baies et racks du datacenter (CMDB)           & 3            & 3            \\
Ouverture de ticket par envoi de mail                     & 3            & 3            \\
Gestion des ressources (certificats, noms de domaine...)  & 3            & 3            \\
KB/FAQ                                                    & 3            & 2            \\
Envoi de mails sur l'information d'activité d'une demande & 3            & 3            \\
Ajout d'informations individuelles sur le portail client  & 3            & 1            \\
Gérer les lieux des clients                               & 3            & 3            \\ \hline
Score total (sur 57)                                      & 51           & 45           \\
\hline
\end{tabular}
\end{table}

\section{Premiers pas dans l'intégration}

Combodo iTop est donc la solution pour laquelle nous avons choisi d'opter. Après en avoir informé mon tuteur par plusieurs mises aux points, je me suis attelé à son installation. Celle-ci devra être faite dans un environnement de pré-production. La dernière marche à passer sera sa mise en production, dans un environnement de production.
\\ \\
J'ai donc commencé la rédaction d'une documentation pour l'installation et l'utilisation de Combodo iTop, utile lorsqu'une autre personne souhaite se pencher sur son fonctionnement en cas de problème ou pour apprendre à mieux l'utiliser. J'y ai notamment documenté le déploiement de sa machine virtuelle, sa configuration. J'ai raccourci les explications en montant un script d'installation complet (interactif, commenté avec gestion d'erreur).
\\ \\
Les intégrations venant par la suite, et voulant faire les choses proprement et ne pas me presser par le temps : j'ai décidé de reprendre le travail de configuration à mon retour pour la troisième période - préférant peaufiner mon travail existant, vu avec mon tuteur.
    \mychapter{Premières activités}{cap:premieres-activites}
\lhead{Premières activités}

Cette deuxième section est dédiée au compte rendu de mes activités et de mon apprentissage faits pendant cette première période dans l'entreprise. Celle-ci intègre ma prise de connaissance du fonctionnement global de la structure, avec l'exposition dans l'état des premiers projets qui m'ont été confiés.  

\section{Prise de marques}

Une grande partie de mon apprentissage pendant ces deux premières semaines s'est dévouée à la considération et à l'intégration du fonctionnement informatique et humain de la structure. Connaître l'environnement logique et opérationnel de son infrastructure est fondamental pour initier une perspective de travail. À contrario de travailler seul, sur son propre équipement, pour son intérêt personnel.

%Avant de pouvoir travailler dans une infrastructure, il est fondamental de connaître son fonctionnement logistique \& opérationnel avant d'initier une procédure de mise en oeuvre.

\subsection{Découverte du fonctionnement de l'infrastructure}

Le service technique comporte les techniciens informatique, les administrateurs systèmes et réseaux \& son directeur technique. Les techniciens informatique sont davantage sollicités pour la manipulation ou l'installation d'appareils chez les clients ou dans les datacenters, ainsi que pour le support et la réparation de matériel.
\\ \\
Les administrateurs élaborent le déploiement de ces équipements, planifient leur maintenance et les administrent depuis le NOC. Le directeur technique orchestrant l'ensemble de ces activités, en plus de travailler comme administrateur systèmes \& réseaux de longue date. Les alternants se formant à prochainement devenir des administrateurs systèmes et réseaux.
\\ \\
Les tâches de chacun vis-à-vis des clients sont définies et expliquées dans des \textbf{bons de travaux}, avec des \textbf{bons de livraison} lorsqu'une installation d'équipement doit être faite. Les devis, facturations, gestion des clients et des prospects sont fait par Mr. Pierre-Sala Éric.
\\ \\
À notre arrivée, des \textbf{fiches de postes} ainsi que des cahiers des charges nous ont été confiés pour nous encadrer dans les projets attendus et notre alternance. Ces fiches sont aussi présentes chez les autres corps de métiers pour encadrer leurs activités.

% dire que ces deux premières semaines on a beaucoup questionné sur les hyperviseurs, le réseau...

%réu exploit + technique; comment l'organisation se fait (prestations, bons de travaux, bons de livraisons, devis, fiches de postes); nos accès; noc; hyperviseurs esxi; réseau d'un "datacenter"; prise de connaissance des bonnes pratiques

\subsection{Accoutumance aux bonnes pratiques}

Chaque entreprise a ses habitudes dans son fonctionnement, via leurs applicatifs ou leur méthodologie (gestion des documentations, de l'archivage, des applicatifs, des sauvegardes)... Cela peut aussi s'appliquer à la nomenclature des systèmes, l'arborescence des fichiers... Il m'a paru essentiel d'assimiler les bonnes pratiques de l'entreprise pour m'y intégrer au mieux : le travail y sera plus agréable pour moi et pour les autres.
\\ \\
Ainsi, à mon arrivée, une fiche d'intégration m'a été distribuée avec mes premiers identifiants de connexion pour les applicatifs communs. J'ai ainsi compris par déduction que le nom des sauvegardes, la rédaction des procédures ou la nomenclature des hôtes étaient réglementées et normalisées : je m'y suis tout de suite adapté.
\\ \\
D'autres domaines comme des astuces ou des coups de pouces n'étaient pas explicités. Je les ai découvert notamment lors d'explications de mon travail aux autres personnes de l'équipe, lorsque celles-ci m'expliquaient comment j'aurais pu simplifier mon travail en effectuant des actions différemment avec certains outils. L'accoutumance à une entreprise passe aussi par une bonne prise en main de ses outils.
\\ \\
Cette accoutumance aux applicatifs, aux astuces de certains logiciels ou autres manières de réfléchir à son travail m'ont beaucoup aidés à prendre mes marques les premières semaines.

\section{Mise en place de services internes}

Pour la première prise en main de l'infrastructure, le cahier des charges demandait l'installation de solutions pour l'amélioration, la simplification ou l'ajout de fonctionnalités au travail général de l'équipe d'ADITU. Ainsi, sans avoir à toucher à la criticité de l'infrastructure des clients, j'ai mis en place plusieurs services internes à ADITU.

\subsection{Aménagement d'un environnement conteneurisé}

Les services reposent sur un environnement conteneurisé Docker. Cela minimise les ressources nécessaire par service, augmente la reproductibilité et l'ensemble des solutions devient plus flexible. Pour simplifier la manipulation des ressources Docker (conteneurs, images, volumes, réseaux), une interface graphique intuitive a été montée, permettant le dépannage rapide des solutions - redémarrage en un clique, diagnostique rapide par indicateurs lumineux...
\\ \\
L'ensemble de cette solution repose sur une machine virtuelle \textit{VM}, hébergée sur un hyperviseur.

\subsection{Installation d'un proxy inverse}

L'ensemble des services ont été montés derrière un proxy inverse \textit{reverse proxy}. Un proxy normal permet la centralisation des points de sortie web pour Internet, pour les faire passer par un uniquement équipement. Utile pour restreindre l'accès à certains site, permettre la mise en tampon de pages web entre utilisateurs, ou pour la rétention d'activités globales sur le web.
\\ \\
Le proxy inverse permet la centralisation des accès et le déploiement de plusieurs services derrière une machine avec la même adresse IP. Le principe est identique au proxy simple, mais dans l'autre sens : au lieu de centraliser les points pour sortir vers Internet, il démultiplexe l'arrivée pour les services voulus. Selon le lien \textit{URL} contacté, le proxy inverse redirige l'activité pour le service voulu, en ayant toutes les URL dirigeant vers la même adresse IP.

\begin{figure}[H]
    \centering
    \includegraphics[width=\textwidth - \textwidth / 6]{Untitled-2024-01-21-1259.png}
    \figurename
    \caption{Vision d'esprit simplifiée du proxy inverse}
    \label{fig:rvproxy}
\end{figure}

\noindent Cette manipulation permet l'hébergement de plusieurs services, par communication sur le même port, en n'utilisant qu'une seule adresse IP publique - précieuse car chères à l'achat. La différenciation du service voulu s'effectuant par l'identification de l'URL renseigné.
\\ \\
Le proxy inverse gère les certificat SSL/TLS pour les services qu'il redirige (la sécurisation du traffic), ainsi que le contact des ports. Les services peuvent être hébergés sur une autre machine, à condition que le proxy inverse puisse la contacter sur le port pour communiquer avec le service hébergé.

% explication utilité reverse proxy pour ce qui va arriver après (une ip publique); schéma; donner le nom de la solution

\subsection{Montage d'un service de partage d'informations sécurisé}

Le premier service monté derrière le proxy inverse fut un service de partage d'informations sécurisé. Celui-ci prend place lors d'échange de mots de passe ou de lignes de configuration avec des clients. ADITU peut en envoyer aux clients comme l'inverse.
\\ \\
Au lieu d'envoyer des mots de passe ou des fichiers de configuration par mails, ceux-ci sont regroupés sur une plateforme accessible par l'attendu uniquement (authentification + autorisation). Cette plateforme permet la non divulgation d'informations sensibles par mail (mots de passe, informations sensibles...), leur chiffrement et un accès sécurisé.
\\ \\
Des mesures de sûreté sont mises en place : possibilité de suppression de l'information après première lecture, confirmation de la lecture, temps limite d'accessibilité à la ressource (utile pour les mots de passe qui "ne doivent pas trainer").
\\ \\
Une solution similaire de secours est aussi mise en place. La première étant traduite en français pour un usage primaire et globale.

%mots de passe et fichier de conf

%expliquer pourquoi besoin de ne pas donner mot de passe en clair dans message (un mec vient, récupère; si self-host, un mec vient, galère et par chance abandonne); envoi de fichiers de configuration par yopass; partage avancés de mots de passe pwdpush (voir si personne à vu et quand, temps maximal, accès une fois)

\subsection{Implémentation d'une solution de partage de fichiers volumineux}

Aucune solution de partage de fichiers volumineux n'était présente chez ADITU. Ainsi, le transport manuel ou par hébergeurs tiers étaient nécessaire pour le transfert de fichiers lourds (fichiers compressés, export de boites mail, enregistrement de caméras...).
\\ \\
Une solution de partage de fichiers volumineux a été monté derrière le proxy inverse pour permettre à ADITU et à ses clients de pouvoir y déposer des fichiers \& de pouvoir les partager. Cette solution est davantage professionnel, plutôt que de passer par des services tiers (Google Drive, Microsoft OneDrive...). De plus, celle-ci est hébergée chez ADITU et bénéficie de la sécurité \& la gestion de son réseau.

\subsection{Déploiement d'une console de vérification de disponibilité}

Le centre de donnée de Dax est certifié ISO 27001 HDS pour l'hébergement de données de santée (pour les hôpitaux...). Cette certification demande des tests périodiques de bascule de réseaux d'opérateurs (accès à Internet) et de restauration de sauvegarde.
\\ \\
Le test de bascule d'opérateur revient à simuler la coupure d'un lien vers Internet pour s'assurer que le datacenter soit toujours accessible depuis l'extérieur en basculant sur un lien de secours - haute disponibilité. Pour s'assurer du bon fonctionnement du test de bascule d'opérateur, une console de vérification de l'état des services a été montée, toujours derrière le proxy inverse.
\\ \\
Cette console vérifie en temps réel la disponibilité des services hébergés dans le datacenter de Dax depuis celui de Bidart. Ainsi, lors de la bascule d'opérateur à Dax, les services sont indisponibles pendant un très court instant pendant que les équipements du datacenter adapte leur configuration pour la nouvelle route - selon l'activité, indistinguable par l'utilisateur.
\\ \\
Le rôle de cette console est de superviser la disponibilité des service hérbergés à Dax, utile lors du test de bascule pour vérifier l'accessibilité des services depuis l'extérieur.

% expliquer pourquoi besoin de se simplifier et pas à avoir à génerer des accès à chaque fois; fichier de conf importants parfois mots de passes dedans ou informations sensibles
    \mychapter{Annexes}{cap:annexes}
\lhead{Annexes}

\section{Résultats d'examens}

\subsection{R4.01 Infrastructures de sécurité}

\begin{table}[!ht]
\Rotatebox{90}{
\centering
\begin{tabular}{|l|l|l|l|l|l|}
\hline
IUT des Pays de l'Adour – DUT RT2 FA 2022-2023 & ~ & ~ & ~ & ~ & ~ \\ \hline
Notes à rendre sur cette liste exclusivement, & ~ & ~ & ~ & ~ & ~ \\ \hline
et à remettre directement au secrétariat & ~ & ~ & ~ & ~ & ~ \\ \hline
Matière : R3.ROM.16 & ~ & ~ & Date : Décembre 2023 & ~ & ~ \\ \hline
~ & ~ & ~ & ~ & ~ & ~ \\ \hline
~ & coeff & 1 & 1 & 0 & ~ \\ \hline
NOMS & Prénoms & Note1 & Note2 & Note3 & Moyenne \\ \hline
ALRIC & LEO-PAUL & 0 & 17 & ~ & 8.5 \\ \hline
DAYON & MATHIEU & 4.5 & 5 & ~ & 4.75 \\ \hline
DEHU & ALEXIS & 15.5 & 18 & ~ & 16.75 \\ \hline
DIENG & Khadim & 9.5 & 8 & ~ & 8.75 \\ \hline
EL AKHAL EL BOUZIDI & MOHAMAD & 7 & 8 & ~ & 7.5 \\ \hline
GORRICHON & MATHIS & 4 & 12 & ~ & 8 \\ \hline
GROLEAU & ANTHONY & 6 & 8 & ~ & 7 \\ \hline
GUIBOREL & TILIO & 16.5 & 12 & ~ & 14.25 \\ \hline
MANAUT & Lilian & 3 & 2 & ~ & 2.5 \\ \hline
MARCHAL & ALEXIS & 5 & 13 & ~ & 9 \\ \hline
MARTIN & BAPTISTE & 4 & 18 & ~ & 11 \\ \hline
MOTZ & MARTIN & 16.5 & 10 & ~ & 13.25 \\ \hline
PETIGAS & ROMEO & 0 & 9 & ~ & 4.5 \\ \hline
PICABIA & ANTTON & 12.5 & 18 & ~ & 15.25 \\ \hline
RECART & ANDONI & 2.50 & 14.00 & ~ & 8.25 \\ \hline
SAUVAGE & CORENTIN & 6.5 & 19 & ~ & 12.75 \\ \hline
~ & MOY & 7.0625 & 11.9375 & \#DIV/0! & 9.5 \\ \hline
\end{tabular}
}
\end{table}

\newpage

\subsection{R4.03 Physique des télécoms}

\begin{table}[!ht]
\Rotatebox{90}{
\centering
\begin{tabular}{|l|l|l|l|l|l|l|}
\hline
IUT des Pays de l'Adour – DUT RT2 FA 2022-2023 & ~ & ~ & ~ & ~ & ~ \\ \hline
Notes à rendre sur cette liste exclusivement, & ~ & ~ & ~ & ~ & ~ \\ \hline
et à remettre directement au secrétariat & ~ & ~ & ~ & ~ & ~ \\ \hline
Matière :R302 & ~ & ~ & Date : 21/12/2023 & ~ & ~ & ~ \\ \hline
~ & ~ & ~ & ~ & ~ & ~ & ~ \\ \hline
~ & coeff & 1 & 1 & 0 & ~ & ~ \\ \hline
NOMS & Prénoms & QCM & Projet & Note3 & Moyenne & PARCOURS \\ \hline
ALRIC & LEO-PAUL & 10,37 & 9.25 & ~ & 9.81 & ROM \\ \hline
DAYON & MATHIEU & 9,27 & 6.38 & ~ & 7.825 & PILPRO \\ \hline
DEHU & ALEXIS & 14,27 & 17.38 & ~ & 15.825 & CYBER \\ \hline
DIENG & Khadim & 12,33 & 10.17 & ~ & 11.25 & CYBER \\ \hline
EL AKHAL EL BOUZIDI & MOHAMAD & 9,83 & 12.88 & ~ & 11.355 & CYBER \\ \hline
GORRICHON & MATHIS & 9,03 & 12.25 & ~ & 10.64 & CYBER \\ \hline
GROLEAU & ANTHONY & 7,93 & 6.96 & ~ & 7.445 & PILPRO \\ \hline
GUIBOREL & TILIO & 10,10 & 10.75 & ~ & 10.425 & ROM \\ \hline
MANAUT & Lilian & 9,77 & 4.61 & ~ & 7.19 & ROM \\ \hline
MARCHAL & ALEXIS & 0 & 5.63 & ~ & 2.815 & ROM \\ \hline
MARTIN & BAPTISTE & 6,57 & 9.08 & ~ & 7.825 & CYBER \\ \hline
MOTZ & MARTIN & 9,73 & 8.22 & ~ & 8.975 & CYBER \\ \hline
PETIGAS & ROMEO & 8,07 & 8.13 & ~ & 8.1 & CYBER \\ \hline
PICABIA & ANTTON & 5,73 & 9.75 & ~ & 7.74 & PILPRO \\ \hline
RECART & ANDONI & 10,77 & 6.07 & ~ & 8.42 & CYBER \\ \hline
SAUVAGE & CORENTIN & 10,73 & 7.35 & ~ & 9.04 & CYBER \\ \hline
~ & MOY & 0 & 9.05375 & \#DIV/0! & 9.0425 & ~ \\ \hline
\end{tabular}
}
\end{table}

\newpage

\subsection{R4.05 Automatisation des tâches d'administration}

\begin{table}[!ht]
\Rotatebox{90}{
\centering
\begin{tabular}{|l|l|l|l|l|l|l|}
\hline
IUT des Pays de l'Adour – DUT RT2 FA 2022-2023 & ~ & ~ & ~ & ~ & ~ & ~ \\ \hline
Notes à rendre sur cette liste exclusivement,  & ~ & ~ & ~ & ~ & ~ & ~ \\ \hline
et à remettre directement au secrétariat & ~ & ~ & ~ & ~ & ~ & ~ \\ \hline
Matière : R3.05 & ~ & ~ & Date : Décembre 2023 & ~ & ~ & ~ \\ \hline
~ & ~ & ~ & ~ & ~ & ~ & ~ \\ \hline
~ & coeff & 2 & 1 & 1 & 2 & ~ \\ \hline
NOMS & Prénoms & Note1 & Note2 & Note3 & Note4 & Moyenne \\ \hline
ALRIC & LEO-PAUL & 15.5 & 17 & 7.5 & 15.5 & 14.42 \\ \hline
DAYON & MATHIEU & 4 & 10 & 2.5 & 7.5 & 5.92 \\ \hline
DEHU & ALEXIS & 13 & 18 & 5 & 17.5 & 14.00 \\ \hline
DIENG & Khadim & 11 & 14 & 5 & 10 & 10.17 \\ \hline
EL AKHAL EL BOUZIDI & MOHAMAD & 5 & 15 & 12 & 12 & 10.17 \\ \hline
GORRICHON & MATHIS & 9 & 9 & 2 & 11 & 8.50 \\ \hline
GROLEAU & ANTHONY & 15 & 8 & 2.5 & 13.5 & 11.25 \\ \hline
GUIBOREL & TILIO & 15 & 7 & 2.5 & 15.5 & 11.75 \\ \hline
MANAUT & Lilian & 4 & 6 & 9 & 10 & 7.17 \\ \hline
MARCHAL & ALEXIS & 9.5 & 15 & 10 & 11 & 11.00 \\ \hline
MARTIN & BAPTISTE & 14 & 13 & 7 & 13 & 12.33 \\ \hline
MOTZ & MARTIN & 12.5 & 16 & 12 & 13 & 13.17 \\ \hline
PETIGAS & ROMEO & 4.5 & 6 & 0.5 & 5 & 4.25 \\ \hline
PICABIA & ANTTON & 3.5 & 6 & 2.5 & 8.5 & 5.42 \\ \hline
RECART & ANDONI & 5.50 & 6.00 & 6.00 & 13.00 & 8.17 \\ \hline
SAUVAGE & CORENTIN & 11 & 15 & 8 & 12 & 11.50 \\ \hline
~ & MOY & 9.5 & 11.3125 & 5.875 & 11.75 & 9.947916667 \\ \hline
\end{tabular}
}
\end{table}

\newpage

\subsection{R4.08 Projet Personnel et Professionne}

\begin{table}[!ht]
\Rotatebox{90}{
\centering
\begin{tabular}{|l|l|l|l|l|l|l|}
\hline
IUT des Pays de l'Adour – BUT RT2 FA 2023-2024 & ~ & ~ & ~ & ~ & ~ \\ \hline
Notes à rendre sur cette liste exclusivement, & ~ & ~ & ~ & ~ & ~ \\ \hline
et à remettre directement au secrétariat & ~ & ~ & ~ & ~ & ~ \\ \hline
Matière :R 306 & ~ & ~ & Date : Décembre 2023 & ~ & ~ \\ \hline
~ & ~ & ~ & ~ & ~ & ~ \\ \hline
~ & coeff & 2 & 0.5 & 0.5 & ~ \\ \hline
NOMS & Prénoms & Théorique & TP1 & TP2 & Moyenne \\ \hline
ALRIC & LEO-PAUL & 4 & 11 & 12 & 6.50 \\ \hline
DAYON & MATHIEU & 3 & 11 & 12 & 5.83 \\ \hline
DEHU & ALEXIS & 15 & 17 & 15 & 15.33 \\ \hline
DIENG & Khadim & 6 & 15 & 10 & 8.17 \\ \hline
EL AKHAL EL BOUZIDI & MOHAMAD & 5 & 15 & 10 & 7.50 \\ \hline
GORRICHON & MATHIS & 4 & 14 & 7 & 6.17 \\ \hline
GROLEAU & ANTHONY & 13 & 15 & 12 & 13.17 \\ \hline
GUIBOREL & TILIO & 8 & 15 & 12 & 9.83 \\ \hline
MANAUT & Lilian & 2 & ABS & ABS & 2.00 \\ \hline
MARCHAL & ALEXIS & 10 & 11 & 15 & 11.00 \\ \hline
MARTIN & BAPTISTE & 12 & 15 & 5 & 11.33 \\ \hline
MOTZ & MARTIN & 11 & 10 & 10 & 10.67 \\ \hline
PETIGAS & ROMEO & 4 & 14 & 7 & 6.17 \\ \hline
PICABIA & ANTTON & 3 & 12 & 15 & 6.50 \\ \hline
RECART & ANDONI & 4.00 & 11.00 & 15.00 & 7.00 \\ \hline
SAUVAGE & CORENTIN & 4 & 15 & 5 & 6.00 \\ \hline
~ & MOY & 6.75 & 13.4 & 10.8 & 8.32 \\ \hline
\end{tabular}
}
\end{table}
    
    % Pós Textuais
    % \nocite{*}
    % \include{pos-textuais/referencias}

\end{document}
