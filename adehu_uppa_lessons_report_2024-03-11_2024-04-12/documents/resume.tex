\begin{resumo}

    Cette deuxième période la plus longue d'enseignements à l'IUT (5 semaines) nous a permis d'aborder un vaste champ de compétences extrêmement utiles et intéressantes selon moi.
    \\ \\
    Nous avons couvert trois modules de notre parcours cybersécurité orienté vers la sécurisation du service DNS avec DNSsec, de la compréhension des attaques utilisées sur des protocoles répandus dans les réseaux locaux d'entreprises et de la compréhension des technologies et pratiques de chiffrement de nos données.
    \\
    Nous avons aussi aussi à manipuler et intégrer des pare-feux physiques dans un réseau local d'entreprise selon leur besoins.
    \\ \\
    Aspect télécommunications, nous avons abordés les réseaux cellulaires 2G, 3G, 4G et d'autres en profondeur, en parallèle avec l'étude physique et mathématique de moyens transmissions modernes d'émission-réception.
    \\ \\
    Un module intéressant était aussi consacré à l'automatisation de nos tâches d'administrations des systèmes et des réseaux. En complément de l'apprentissage d'un anglais professionnalisé et d'un travail sur notre communication.
    
    \palavrasChave{Sécurisation; Protocoles; ARP; ICMP; DNS; Chiffrement; Cryptographie; Clés; Chiffrement symétrique; Asymétrique; Intégrité; Confidentialité; Authentification; Authenticité; Signature; Certificat; Fonctions mathématiques et algorithmiques; OFDM; COFDM; MIMO; CDMA; TDMA; FHSS; Réseaux cellulaire; Filtres numériques; Filtres analogiques}
\end{resumo}