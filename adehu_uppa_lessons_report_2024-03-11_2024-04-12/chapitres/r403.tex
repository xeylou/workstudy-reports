\renewcommand{\figurename}{}
\mychapter{R4.03 Physique des télécoms (27h)}{cap:r403} 
\lhead{R4.03 Physique des télécoms (27h)}

\vspace*{0.2cm}%
      \large
      \href{}{\color{black}Enseignant\\M. Christophe Baillot}\\%
      \normalsize
\vspace*{0.5cm}%

Pour ce deuxième module exclusif à la physique des télécoms, nous avons abordés des notions avancées en électronique touchant globalement aux réseaux cellulaires dans les télécoms, faisant sens à notre module R4.04 et R4.02.
\\ \\
Nous y avons vu les émetteurs et récepteurs superhétérodynes et 0 IF, de la modulation multi-porteuse OFDM

\section{Les systèmes de transmission étudiés}

Nous avons étudié les systèmes de transmission superhétérodynes, en soit des systèmes de transmission modulant deux fois le signal sur une fréquence porteuse et sur une fréquence intermédiaire, cette dernière pour le transport.
\\ \\
Leur intérêt est de pouvoir recentrer le signal sur une fréquence particulière tout en conservant la modulation permise par la fréquence porteuse. À la réception, remultiplier le signal reçu par la même fréquence intermédiaire permettra avec un filtre de récupérer le signal transporté avec transposition.
\\ \\
Le récepteur 0 IF, pour aucune fréquence intermédiaire (ou image), permet de récupérer le signal modulé en bande de base en échantillonnant de telle sorte que le repliement de spectre permette que le signal reçu s'amplifie en \texttt{0 Hz}.
\\ \\
Ce dernier est utilisant à très hautes fréquences, les systèmes actuels ne pouvant échantillonner des valeurs arrivant aussi rapidement (à hautes fréquences).

Ainsi, nous avons vu successivement plusieurs technologies, chacune répondant petit à petit à des problématiques grace aux avancées faites

\subsection{Compréhension des principes liés au MIMO}

Nous avons finalement abordé les technologies MIMO et MISO, utilisant des émetteurs à plusieurs antennes, et/ou non des récepteurs à plusieurs antennes aussi.
\\ \\
Leur utilisation permette de mieux recevoir le signal, et d'en diminuer le rapport que l'on en voit avec le bruit. Ou d'envoyer une information d'antenne à antenne quand le nombre est réciproque des deux côtés (en décorélant les flux, envoyer X fois plus d'informations où X le nombre d'antennes).