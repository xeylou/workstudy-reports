\renewcommand{\figurename}{}
\mychapter{R4.07 Expression Communication 4 (15h)}{cap:r406} 
\lhead{R4.07 Expression Communication 4 (15h)}

\vspace*{0.2cm}%
      \large
      \href{}{\color{black}Enseignant\\M. Walter Rohrig}\\
      \normalsize
\vspace*{0.5cm}%

% contrôle théorique sur les enseignements
% mise en situation de résolution d'un prolblème

Ayant pour objectif d'améliorer notre communication en entreprise, ce module s'est vu être enrichissant en connaissances théoriques et pratiques pour communiquer plus efficacement. Pour ce faire, nous avons été évalués de manière théorique sur notre compréhension des notions abordées pour au mieux mener une conversation juste avec intérêt, et en pratique sur un atelier de résolution de conflits en entreprise.
\\ \\
L'apprentissage théorique englobait un ensemble de notions comme le droit à l'erreur, la communication interne à l'entreprise comme externe avec des clients, partenaires... mais aussi la gestion du circuit de communication et les prémices de la gestion d'un conflit.
\\ \\
Sur cette dernière notion, nous avons aussi été mis en situation pour gérer un conflit inter sites d'une entreprise. En effet, nous devions avec les tenants d'un conflit trouver un moyen de cerner les attendus pour proposer un plan logique de communication et de gestion du conflit.