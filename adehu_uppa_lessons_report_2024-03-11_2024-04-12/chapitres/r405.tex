\renewcommand{\figurename}{}
\mychapter{R4.05 Automatisation des tâches d'administration (7h30)}{cap:r405} 
\lhead{R4.05 Automatisation des tâches d'administration (7h30)}

\vspace*{0.2cm}%
      \large
      \href{}{\color{black}Enseignant\\M. Philippe Arnould}\\%
      \normalsize
\vspace*{0.5cm}%

% préciser que repris prochaine période


Initié cette période mais finalisé celle suivante, ce module abordait l'automatisation de nos tâches d'administration systèmes et réseaux. Ainsi, nous avons montés des infrastructures sur lesquels nous avons utilisé des API, des scripts ou du code pour effectuer en une manipulations des actions qui auraient été répétitives et longues si faites manuellement sur chaque équipement.
\\ \\
En conséquent, nous pouvions déployer nos actions sur plusieurs machines simultanément, sans avoir à manuellement répliquer chaque configuration, préparer une sauvegarde, sauvegarder la configuration manuellement à chaque passage...

\section{Prise en main d'outils d'automatisation}

Plusieurs outils nous ont permis d'automatiser nos tâches : des API et des intégrations. Nous avons majoritairement pris en main Ansible et paramiko pour nos manipulations.

% netconf, restconf, python, ansible, chef, puppet

\section{Principde de fonctionnement des API}

Les API sont des interface exposées à l'extérieur d'une machine, que l'on peut appeler pour lui demander d'effectuer une ou plusieurs manipulations en interne. Ainsi, en donnant une suite d'instruction à faire à une API, celle-ci pourra l'appliquer à l'équipement et nous en renvoyer le résultat.
\\ \\
L'intérêt des API est de pouvoir être appelées par différent supports (différents langages de programmations, solutions, programmes, humains ou non) et de permettre une réplication de l'instruction à d'autres équipements (un programme va contacter 100 équipements où il faut changer une information).

\subsection{Intégrations que nous en avons fait}

Nous avons aussi manipuler des intégrations de protocoles que nous utilisions déjà. Nous avions l'habitude d'ouvrir une session de connexion à distance SSH à nos machines pour leur renseigner des actions à effectuer pour changer leur état. Aujourd'hui, nous pouvons générer du code informatique qui va lui-même aller le renseigner à différentes machines et nous renvoyer le résultat.
\\ \\
Ceci permet de ne pas avoir à passer 50, 100 machines manuellement pour répéter le même processus. Des outils comme \texttt{paramiko} ou \texttt{netmiko} automatise nos tâches sur SSH par exemple. Sans oublié les indispensables \texttt{Ansible} (beaucoup utilisé), \texttt{Puppet} ou \texttt{Cheff} pour la gestion d'un parc, qui peuvent faire appel à des applications comme paramiko et netmiko.
\\ \\
Des notions plus avancées et standardisées peuvent être amenées à être utilisés mais non abordés à l'IUT, \texttt{netconf} (ssh) et \texttt{restconf} (http). Ceux-ci sont des protocoles dédiés à la standardisation d'accès aux informations de tout équipement confondu, de pouvoir changer leur état sans changer les instructions qu'on donne pour chaque constructeur... Tout cela dans le modèle d'application \texttt{YANG}, qui donne la structure à utiliser pour effectuer des appels à consultation ou modifications. Souvent couplé au modèle \texttt{Agile} qui définit les bonnes pratiques à avoir lors d'appels à ces protocoles (sauvegarde, tests de vérification après application, rédaction d'un rapport...).