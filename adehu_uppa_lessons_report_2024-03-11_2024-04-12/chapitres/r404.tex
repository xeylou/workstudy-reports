\renewcommand{\figurename}{}
\mychapter{R4.04 Réseaux cellulaires (27h)}{cap:r404} 
\lhead{R4.04 Réseaux cellulaires (27h)}

\vspace*{0.2cm}%
      \large
      \href{}{\color{black}Enseignant\\M. Yannick Lespine   }\\%
      \normalsize
\vspace*{0.5cm}%

Grand module dans les télécommunications, celui-ci nous a énormément appris sur les réseaux cellulaires notamment pour la téléphonie mobile. Reprenant des enseignements sur les antennes et la physique permettant le transport d'informations, le module R4.04 nous a plongé dans la compréhension des réseaux mobiles 2G, 3G et 4G; avec des parties applicables au Wi-Fi.
\\ \\
Est entendu réseau cellulaire un réseau couvrant une surface, découpé en petites zones alors dites cellules pour couvrir l'ensemble de la surface.
\\ \\
Ce principe est utilisé dans la téléphonie mobile (3G, 4G...). Ainsi, vous avez des antennes partout en France permettant de couvrir globalement le territoire. Les antennes couvrent une cellule.
\\\\
Appliqué à ceci, nous avons vu l'intérêt des pylônes pour les opérateurs et pourquoi sont-ils si haut. Nous avons aussi vu les ondes et fréquences délimitant les normes (GSM, 3G, 4G...), les technologies que celles-ci utilisent et ce très en profondeur.
\\ \\
Ainsi, nous avons vu en détails toutes les procédures de communication entre un appareil mobile, un pylône et le réseau, pour transférer de la données ou de la voix et les problèmes que cela engendre sur de grandes distances (chemins multiples). Nous avons notamment étudié comment un téléphone se raccorde à une antenne, comment le faire sonner, que ce passe-t-il lorsqu'il est en déplacement (voiture ou autre) et que l'on est en communication.
\\ \\
En travaux pratique, à l'aide d'analyseur de spectres : nous avons étudiés les signaux reçus des antennes aux alentours et comparé nos résultats avec nos connaissances. Nous avons aussi étudié la couverture réseau entre les villes de Mont-de-Marsan et Dax, les intérêts des opérateurs selon tels pylônes, couvertures...